\section{Поиск критической точки модели IsingISAW на треугольной решётке}

В данном разделе проводится исследование критической области модели Изинга на случайном булжании без самопересечений (далее, IsingISAW) на треугольной решётке.
В отличие от классической модели на квадратной решётке, узлы треугольной решётки имеют две дополнительные связи по одной из диагоналей (см. рисунок \ref{fig:lattices}), 
вследствие чего координационное число данной модификации (кол-во возможных связей у одного узла) увеличено по сравнению с квадратной с 4 до 6.

