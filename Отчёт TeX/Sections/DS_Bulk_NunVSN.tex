\subsection{Итоговое сравнение функций долей узлов $n_i$}
\label{sec:n_final}

Данный подраздел завершает сравнение функций $f_i$ \eqref{eq:n_i_log_log} и $g_i$ \eqref{eq:n_i_u_log_log}, на этот раз напрямую оценивая их сходство по линейным, степенным и свободным коэффициентам соответственно.

В первую очередь сравним свободные коэффициенты шкалирующих функций \eqref{eq:n_i_log_log}, \eqref{eq:n_i_u_log_log} (cтолбцы $b$ и $d$ таблиц \ref{tab:n_i_log_log} и \ref{tab:n_i_u_log_log} соответственно) построчно. Очевидно, несмотря на разные структуры функций ($f_i$ - функция от $N$, $g_i$ - фактически сложная функция от $N$), пределы функций $f_1$ и $g_1$, $f_2$ и $g_2$ и т.д. должны полностью совпадать, так как $\Nun \to \infty$ при $N \to \infty$. Так же, в виду обнаруженного степенного характера шкалирования величин при бесконечном количестве шагов, интересно различие между ближайшими к пределу результатами симуляций Монте-Карло (последняя строка таблицы \ref{tab:Ran_Walk_neigh}) и полученными пределами. 

\begin{table}[h]
\centering
\begin{tabular}{|c|c|c|c|}
\hline
 & b & d & $n_i(N=10000)$ \\ \hline
$n_1$ & 0.014(1) & 0.015(1) & 0.02179(1)\\ \hline
$n_2$ & 0.037(2) & 0.053(2) & 0.15628(6)\\ \hline
$n_3$ & 0.202(3) & 0.203(2) & 0.27991(7)\\ \hline
$n_4$ & 0.759(5) & 0.741(5) & 0.5420(1)\\ \hline
\end{tabular}
\caption{Пределы шкалирующих функций долей узлов (с фиксированным числом соседей, построчно) от $N$ \eqref{eq:n_i_log_log} (левый столбец), от $\Nun$ \eqref{eq:n_i_u_log_log} (центральный столбец) и ближайшие к пределу результаты Монте-Карло (правый столбец).}
\label{tab:bd_compare}
\end{table}

Наглядное сравнение проведено на таблице \ref{tab:bd_compare}. 
Видно, что лишь у двух величин из четырех свободные коэффициенты шкалирующих функций равны между собой в пределах погрешности - это пределы долей $n_1$ ($b_1$ и $d_1$) и $n_3$ ($b_3$ и $d_3$). 
Другие два предела значительно отличаются -- их разница многократно превышает погрешность.
Однако между функциями сохраняется соразмерность пределов: как $b_1 < b_2 < b_3 < b_4$, так и $d_1 < d_2 < d_3 < d_4$.


Так же особо примечательна огромная разница между результатами симуляций Монте-Карло и соответствующими пределами у всех величин.
Это показывает значительное влияние степенной составлящей функций вблизи $1/N \to 0$.

\begin{table}[h]
\centering
\begin{tabular}{|c|c|c|}
\hline
 & a & s \\ \hline
$n_1$ & 0.417(2) & 0.479(2) \\ \hline
$n_2$ & 0.171(1) & 0.214(1) \\ \hline
$n_3$ & 0.219(3) & 0.244(2) \\ \hline
$n_4$ & 0.189(3) & 0.225(4) \\ \hline
\end{tabular}
\caption{Степенные коэффициенты шкалирующих функций долей узлов (с фиксированным числом соседей, построчно) от $N$ \eqref{eq:n_i_log_log} (левый столбец) и от $\Nun$ \eqref{eq:n_i_u_log_log} (правый столбец).}
\label{tab:as_compare}
\end{table}

Проведём аналогичное сравнение степенных коэффициентов $a$ и $s$ шкалирующих функций \eqref{eq:n_i_log_log}, \eqref{eq:n_i_u_log_log} относительно соответствующей наблюдаемой величины на таблице \ref{tab:as_compare}. 
Численная схожесть степенных коэффициентов функций не наблюдается ни при каких величинах $n_i$.
Заметно, что коэффициенты функций $g_i$ несколько больше чем у функций $f_i$.
Из этого можно справедливо предположить, что число уникальных узлов $\Nun$ обладает более сильными геометрическими свойствами, чем количество шагов $N$. 
Это так же подтверждается результатами шкалирования доли уникальных узлов (строка $\nun$ таблицы \ref{tab:n_i_log_log}) -- доля уникальных узлов случайного бесконечно долгого блуждания очень небольшая, чуть больше $16\%$.
Выходит, в бесконечно отдалённый момент времени блужданию нужно совершить в среднем больше 6 шагов, чтобы найти не посещённый ранее узел решётки.

Сохраняется соразмерность степенных коэффициентов: как $a_1 > a_3 > a_4 > a_2$, так и $s_1 > s_3 > s_4 > s_2$. 
Примечательно так же то, что у величин с наименьшими степенными коэффициентами среди всех четырех ($n_2, n_4$) имеют разные пределы функций $f_i$ и $g_i$ - можно предположить, что это вызвано техническими особенностями метода наименьших квадратов, использованного для оценки коэффициентов шкалирующих функций.
Малое значение степеного коэффициента шкалирующей функции означает слабое движение антиградиента функции ошибок.
Это даёт понимание, что фактически погрешность пределов функций $f_i$ и $g_i$ больше, чем была подсчитана методом ранее. 


\begin{table}[h]
\centering
\begin{tabular}{|c|c|c|}
\hline
 & k & q \\ \hline
$n_1$ & 0.3425(8) &  0.313(1) \\ \hline
$n_2$ & 0.573(4) & 0.567(3) \\ \hline
$n_3$ & 0.588(3) & 0.542(5) \\ \hline
$n_4$ & -1.239(9) & -1.20(1) \\ \hline
\end{tabular}
\caption{Линейные коэффициенты шкалирующих функций долей узлов (с фиксированным числом соседей, построчно) от $N$ \eqref{eq:n_i_log_log} (левый столбец) и от $\Nun$ \eqref{eq:n_i_u_log_log} (правый столбец).}
\label{tab:kq_compare}
\end{table}

Сравнение линейных коэффициетов изображено на таблице \ref{tab:kq_compare}. 
В данном случае между функциями не наблюдается ни численного, ни соразмерного сходства - несмотря на равенство всех коэффициентов до первого знака после запятой, в пределах погрешности равенства нет.

Имеется лишь знаковое сходство -- все коэффициенты функций, кроме $k_4$ и $q_4$, положительны, что говорит о сохранении схожести функций по поведению.

В итоге, с некоторым допущением относительно используемых методов апроксимации, мы получили примерное равенство пределов шкалирующих функций от $N$ и от $\Nun$, что говорит о корректности полученных функций шкалирования.


\newpage