\section{Свободная энергия}
Учитывая предыдущие вычисления, будет удобно проверить формулу другой величины - свободной энергии дл 
я случая h = 0. Для этого слегка преобразуем нашу статсумму:
\[ Z\obc = \lp^{N-1}A_{+}(1 - (\frac{\lm}{\lp})^{N-1}(\frac{A_{-}}{A_{+}}) \]

где \[ A_{+} = \frac{e^{\bj} \sinh{\bh}^2}{Q} + \frac{1}{e^{\bj}Q} +  \cosh{\bh}\]
\[ A_{-} = \frac{e^{\bj} \sinh{\bh}^2}{Q} + \frac{1}{e^{\bj}Q} -  \cosh{\bh}\]

Тогда свободная энергия для случая h=0 будет равна:
\[ F_{h=0} = k_{B}T\ln{Z} =  k_{B}(N-1)\ln{\lp} + k_{B}T\ln{A+} + k_{B}T\ln{(1 - (\frac{\lm}{\lp})^{N-1}(\frac{A_{-}}{A_{+}}))}\]

Ранее мы узнали все преобразования при h = 0: $A_{+} = 2, A_{-} = 0$, следовательно:
\[ F_{h=0} = k_{B}T(N-1)\ln{(2\cosh{\bj})} + k_{B}T\ln{2}\]

Результаты снова совпали с формулой из учебника \cite{Swen}.

Тогда руководствуясь предыдущими расчётами для случая J=0, свободная энергия для данного случая (зная, что $\lp = 2\cosh{\bh}, \lm = 0, A_{+} = 2\cosh{\bh}$) будет равна:
\[ F_{J=0} = k_{B}T N \ln{(2\cosh{\bh})}\]