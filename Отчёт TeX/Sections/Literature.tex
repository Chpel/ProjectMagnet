\section{Литературный обзор}

С целью поиска информации о локальном координационном числе (что в случае блужданий может также быть названо числом соседей узла), был проведён обзор литературы, возможно имеющей отношение к рассматриваемым в рамках проекта моделей.

\subsection{Livne, Meirovich: Polymers Adsorbed on a surface}

\subsubsection{Особенности модели блуждания}

В работе \cite{LivneSAW1988} исследуется поведение адсорбирующего случайного блуждания без самопересечений на кубической решётке со следующими особенностями симуляции

\begin{itemize}
    \item Случайное блуждание длины N+1  строится пошагово (N+1 мономеров в цепочке или N шагов), из начала координат (x=0, y=0, z=0) с ограничением на верхнее полупространство (то есть, z >= 0 и плоскость z=0 имеет открытые граничные условия).
    \item Энергия конформации считается как число мономеров, лежащих на поверхности (у которых $z_{i} = 0$), умноженное на константу взаимодействия полимера и поверхности $\epsilon$
    \item Вероятность i-й конформации считается последовательно: вводится новая статсумма, суммирующая для заданного направления текущей недостроенной цепочки всевозможные хвосты остаточной длины (10)\cite{LivneSAW1988}. 
\end{itemize}

\subsubsection{Подробнее о статсумме и методе Сканирования }

В данном подразделе вольным образом объясняется действие статсуммы, созданное методом сканированния. Так как при симуляции строится новое блуждание "c нуля", требуется оценка вероятности как каждого шага (точнее, направления $v_{k}$) так и всего блуждания.

Поэтому для k-го шага вероятность рассчитывается следующим образом:

\begin{enumerate}
    \item Считается статсумма куска будущего блуждания из b ($<= N - k + 1$) шагов, начинающая с направления v на высоте $z_{k-1}$:
    
    \begin{equation}
        Z_{k}(v, b, z_{k-1}, v_{k-1}) = \sum_{j}\exp{(-\epsilon m_{j}(0)/k_{b}T)}
        \label{Z_Lenvi}
    \end{equation}
     
    \item Затем проводится расчёт вероятности выбрать направление v из всех возможных на k-м шаге:
    
    \begin{equation}
        p_{k}(v|b,z_{k-1},v_{k-1}) = Z_{k}(v, b, z_{k-1}, v_{k-1}) / \sum_{v} Z_{k}(v, b, z_{k-1}, v_{k-1})
        \label{p_k_Lenvi}
    \end{equation}
    
    \item Итоговой вероятностью всего построения будет произведение всех вероятностей каждого шага по выбранным направлениям:
    
    \begin{equation}
        P_i(b) = \prod_{k=1}^{N} p_{k}(v_{k}|b,z_{k-1},v_{k-1})
    \end{equation}
\end{enumerate}

\subsubsection{Результаты работы}

Основными итогами работы являлось подтверждение эффективности метода "сканирования" для работы с длинными цепочками в модели адсорбирующего блуждания, определено критическое шкалирование перпердикулярного радиуса инерции (радиуса инерции проекции блуждания на ось z), а также профиля мономерной концентрации $p(z)$ (средняя доля узлов конформации длины N+1 на фиксированной высоте z от поверхности).

Информации о локальном координационном числе в статье найдено не было.

\newpage

\subsection{Madras, Sokal: The Pivot Algorithm}

Работа \cite{madras1988pivot} повествует о работе и эффективности алгоритма Пивота в изучении модели случайного блуждания без самопересечений (СБС).

\subsubsection{Основные принципы алгоритма}

Каждый шаг алгоритма проводит следующие действия над уже сгенерированной цепочкой длины N+1:

\begin{itemize}
    \item Случайно выбирается с равномерным распределением для рассматриваемых узлов $p_{k} = 1/N$ k-й узел цепочки ($0 <= k <= N-1$, хотя начальную точку k=0 на практике не используют)
    \item Последующую половину цепочки ($\omega_{k+1}, \omega_{k+2},\dots,\omega_{N}$ заменяют элементов группы симметрии (проще говоря, отражают, поворачивают или проводят комбинацию этих действий)
    \item В случае, если полученная операцией цепочка осталась без самопересечений, шаг принимается - в противном случае, шаг производится заново
\end{itemize}

В статье так же была доказана эргодичность алгоритма, а так же средние вероятности принятия каждого из возможных преобразований.

Для симуляций в качестве стартовой позиции использовалось два варианта: прямые цепочки ''rods'', при которых проволилось некоторое кол-во шагов до достижения термального равновесия системы (в таком состоянии процесс из следующих состояний цепочки становится близким по расспределению к стационарному стохастическому), или же ''димеризованные цепочки'' , состояние которых уже считается равновесным. Второй метод становится крайне времезатратным при большой длине цепочки, поэтому при N>=2400 чаще применялась термолизация прямых цепочек.

Пристальное внимание в статье было обращено к среднему радиусу инерции $S^{2}_{N}$ и квадрату расстояния между концами $\omega^{2}_{N}$, а так же к оценке метрической экспоненты $\upsilon$, характеризующей обе величины в крит. области модели: 

\begin{align*}
    \la \omega^{2}_{N} \ra &\sim N^{2\upsilon} \\
    \la S^{2}_{N} \ra &\sim N^{2\upsilon} 
\end{align*}

В оценке будущей работы было так же отмечено, что алгоритм Пивота не подходит для расчёта связующей $\mu$ и критической $\gamma$ экспонент (связующую константу так же называют \textit{эффективным координационным числом}), так как алгоритм алгоритм работает лишь в случае канонического ансамбля (при фиксированной длине цепочки) и требуется алгоритм, работающий уже в большом каноническом ансамбле (с цепочками изменяемой длины).

В статье не рассматривалось как таковое ''число соседей узлов''.


\subsection{Спицер, Основные принципы случайного блуждания, глава 3}

Данный подраздел посвящён рассмотрению случая двумерного возвратного случайного блуждания - блуждания, движущемся по состояниям $R$ до достижения одного из элементов $A \subset R$. Под $T$ или $T_A$ мы будем подразумевать момент остановки - минимальное число $1<= k <= \infty$, такое что $x_k \in A$, то есть минимальное время достижение процессом {x_i} состояния из пространства A.

\subsubsection{Основные вероятностные функции}

Здесь будут более тщательно описаны используемые в главе функции вероятностей перехода.

$Q_n(x,y)$ определена на $(R-A) \times (R-A),\ \ n >= 0$ и обозначает вероятность попасть на n-м шаге попасть в $y$ (при $x_0 = x$), не попав за это время в A. Логично, что при остановке $T<n$ вероятность достижения на n-м шаге не существует, т.к. проццесс остановлен.

\begin{equation}
 Q_n(x,y) = P_x[x_n=y; T>n]
\end{equation}

Функция $H^{(n)}_A(x,y)$, наоборот, определяет вероятность n-м шаге остановиться в $y \in A$ (то есть, $y$ является первым состоянием из $A$, в которое попал процесс. В данном случае $H_A$ определено на $R \times A$

\begin{equation}
H^{(n)}_A(x,y) = 
	\begin{cases}
		P_x[x_T=y; T=n], \ \ \ x \in R-A \\
		0, \ \ \ x \in A, n>=1 \\
		\delta(x,y), \ \ \ x \in A, n=0
	\end{cases}
\end{equation}

$H_A(x,y)$ является обобщением предыдущей функции по времени, определяя лишь вероятность остановки процесса, начавшегося в $x$, в $y \in A$ и определена там же как и $H^{(n)}_A(x,y)$.

\begin{equation}
H_A(x,y) = 
	\begin{cases}
		P_x[x_T=y; T < \infty], \ \ \ x \in R-A \\
		\delta(x,y), \ \ \ x \in A
	\end{cases}
\end{equation}

Для случая $x \in R-A$ эту функцию можно определить так же как:

\begin{equation}
H_A(x,y) = \sum^{\infty}_{n=0}H^{(n)}_A(x,y)
\end{equation}

Особым случаем является вероятность $\Pi_A(x,y)$, существование которой обусловлено тем фактом, что время остановки должно быть натуральным числом - строго говоря, процесс может начатся в $x \in A$, пройти по ${x_1, x_2,...x_T-1 \in R-A}$ и остановиться в $y \in A$.

\begin{equation}
\Pi_{A}(x,y) = P_x[x_T=y, T < \infty]
\end{equation}

Последняя функция - $g_A(x,y)$, обобщает  по времени $Q_n$:

\begin{equation}
g_A(x,y) = 
	\begin{cases}
		\sum^{\infty}_{n=0}Q_n(x,y), \ \ \ x, y \in R-A \\
		0, otherwise
	\end{cases}
\end{equation}

Из общих понятий нам также понадобится $G(x,y)$ - ожидаемое число попаданий в $y$ при начальной точке $x$:

\begin{equation}
G(x,y) = \sum_{n=0}^{\infty}P_x[x_n=y]
\end{equation}

\subsubsection{Соотношения между функциями}

Перейдём к некоторым предложениям из книги, которые позволят более полно понять природу некоторых функций в зависимости от начального состояния в них. Здесь будет описана лишь их вольная интерпретация, без доказательства.

Пункт (а) предложения 1 проводит важную связь между $\Pi_A(x,y)$ и $H_A(x,y)$ при разных начальных состояниях: при $x \in R-A$ выражение равно нулю, как так оба слагаемых выражают один и тот же процесс из начального состояния до множества остановки, как со смещением (первое слагаемое), так и без него (правое). Равенство для случая $x \in A$ подтверждает раннюю интерпретацию функции $\Pi_A(x,y)$: шаг из множества остановки (P(x,t)) и затем движение из t до остановки снова и $A$.




