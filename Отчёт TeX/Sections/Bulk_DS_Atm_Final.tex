\subsection{Сравнение результатов вероятностей атмосфер от $N$ и $\Nun$}

Проведём, аналогично разделу \ref{sec:n_final}, численное сравнение коэффициентов шкалирующих функций $\pkn(N)$ и $\pkn(\Nun)$ из таблиц \ref{tab:p_i_log_log} и \ref{tab:p_i_u_log_log}.

\begin{table}[h]
\centering
\begin{tabular}{|c|c|c|c|}
\hline
 & b & d & $p^{(i)}(N=10000)$ \\ \hline
$p^{(0)}$ &  0.62(1) & 0.59(2) &  0.436369\\ \hline
$p^{(1)}$ & 0.213(6) & 0.214(6) & 0.229075\\ \hline
$p^{(2)}$ & 0.137(4) & 0.141(3) & 0.185300\\ \hline
$p^{(3)}$ & 0.092(4) & 0.097(3) & 0.149256\\ \hline
\end{tabular}
\caption{Пределы шкалирующих функций вероятностей блужданий с фиксированной атмосферой (от 0 до 3, сверху вниз) от $N$ \eqref{eq:p_i_log_log} (левый столбец), от $\Nun$ \eqref{eq:p_i_u_log_log} (центральный столбец) и ближайшие к пределу результаты Монте-Карло из таблицы \ref{tab:randw_p_atm} (правый столбец).}
\label{tab:bd_atm_compare}
\end{table}

Как видно по таблице \ref{tab:bd_atm_compare}, все пределы шкалирующих функций \eqref{eq:p_i_log_log}, \eqref{eq:p_i_u_log_log} при $1/N \to \infty, 1/\Nun \to \infty$ равны в пределах погрешности.
Так же, как в случае с пределами долей узлов $n_i$ (таблица \ref{tab:bd_compare}), наблюдается сильное различие между ближайшими к пределу результатами симуляций \ref{tab:randw_p_atm}.
Это показывает сильное влияние степенного шкалирования результатов вблизи нуля аргументов функций: чем меньше степенной коэффициент ($a$ и $s$ для $\pkn(N)$ и $\pkn(\Nun)$ соотвественно), тем больше абсолютная разница между пределом и ближайшим экспериментом.

\begin{table}[h]
\centering
\begin{tabular}{|c|c|c|}
\hline
 & a & s \\ \hline
$n_1$ & 0.417(2) & 0.479(2) \\ \hline
$n_2$ & 0.171(1) & 0.214(1) \\ \hline
$n_3$ & 0.219(3) & 0.244(2) \\ \hline
$n_4$ & 0.189(3) & 0.225(4) \\ \hline
\end{tabular}
\caption{Степенные коэффициенты шкалирующих функций долей узлов (с фиксированным числом соседей, построчно) от $N$ \eqref{eq:n_i_log_log} (левый столбец) и от $\Nun$ \eqref{eq:n_i_u_log_log} (правый столбец).}
\label{tab:as_atm_compare}
\end{table}

\begin{table}[h]
\centering
\begin{tabular}{|c|c|c|}
\hline
 & k & q \\ \hline
$n_1$ & 0.3425(8) &  0.313(1) \\ \hline
$n_2$ & 0.573(4) & 0.567(3) \\ \hline
$n_3$ & 0.588(3) & 0.542(5) \\ \hline
$n_4$ & -1.239(9) & -1.20(1) \\ \hline
\end{tabular}
\caption{Линейные коэффициенты шкалирующих функций долей узлов (с фиксированным числом соседей, построчно) от $N$ \eqref{eq:n_i_log_log} (левый столбец) и от $\Nun$ \eqref{eq:n_i_u_log_log} (правый столбец).}
\label{tab:kq_atm_compare}
\end{table}


\newpage

\subsection{Общее сравнение поведений атмосфер блужданий и долей узлов простого случайного блуждания}

Для простого случайного блуждания можно отметить сильное по смыслу родство понятий "атмосферы k" блуждания и "доли узлов с i соседями". 
В данном случае очевидно, что если у конца блуждания некоторое число $v$ соседей, то количество незанятых вокруг него узлов всегда равно $4-v$. 
Связь этих свойств гораздо сильнее, чем в случайном блуждании без самопересечений, где для попытки их сопоставления требовалось доп. условие, что блуждание не замкнуто и всегда имеет возможность добавить к себе доп. узел.

Рассмотрим таблицы \ref{tab:n_i_log_log} и \ref{tab:p_i_log_log}, \ref{tab:n_i_u_log_log} и \ref{tab:p_i_u_log_log} на предмет сходства коээфициентов между функциями $\la n_v \ra$ и $\la p^{(4-v)} \ra$.
Сравнение показывает, что между таблицами отсутсвует явная корреляция, за исключением идентичности знаков линейного коээфициента фитирующих функций как от $N$, так и от $N_{unique}$. 

Можно предположить, что связь между ними существует - об этом говорит как подтверждённый степенной характер сходимости, так и схожесть знаков линейных коээфициентов - однако, она крайне слаба ввиду разной статистической мощности наблюдаемых величин - очевидно, что $\la n_v \ra$ охватывает геометрическое поведение всего блуждания, а $\la p^{(v)} \ra$ описывает поведение лишь на его концах, характер которых с увеличением длины блуждания становится некоррелируемым с поведением внутренних узлов.

\newpage

\subsection{Планируемая деятельность}

\begin{itemize}
\item 3-я итерация программного кода для симуляции модели Rand-Walk - добавление в модель аналога квадратной решётки с целью упрощения расчётов уникальных узлов и их соседей.
\end{itemize}
