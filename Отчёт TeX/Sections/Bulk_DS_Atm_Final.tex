\subsection{Сравнение результатов вероятностей атмосфер от $N$ и $\Nun$}

Проведём, аналогично разделу \ref{sec:n_final}, численное сравнение коэффициентов шкалирующих функций $\pkn(N)$ и $\pkn(\Nun)$ из таблиц \ref{tab:p_i_log_log} и \ref{tab:p_i_u_log_log}.

\begin{table}[h]
\centering
\begin{tabular}{|c|c|c|c|}
\hline
 & b & d & $p^{(i)}(N=10000)$ \\ \hline
$p^{(0)}$ &  0.62(1) & 0.59(2) &  0.436369\\ \hline
$p^{(1)}$ & 0.213(6) & 0.214(6) & 0.229075\\ \hline
$p^{(2)}$ & 0.137(4) & 0.141(3) & 0.185300\\ \hline
$p^{(3)}$ & 0.092(4) & 0.097(3) & 0.149256\\ \hline
\end{tabular}
\caption{Пределы шкалирующих функций вероятностей блужданий с фиксированной атмосферой (от 0 до 3, сверху вниз) от $N$ \eqref{eq:pi_N} (левый столбец), от $\Nun$ \eqref{eq:pi_Nu} (центральный столбец) и ближайшие к пределу результаты Монте-Карло из таблицы \ref{tab:randw_p_atm} (правый столбец).}
\label{tab:bd_atm_compare}
\end{table}

Как видно по таблице \ref{tab:bd_atm_compare}, все пределы шкалирующих функций \eqref{eq:pi_N}, \eqref{eq:pi_Nu} при $1/N \to \infty, 1/\Nun \to \infty$ равны в пределах погрешности.
Так же, как в случае с пределами долей узлов $n_i$ (таблица \ref{tab:bd_compare}), наблюдается сильное различие между ближайшими к пределу результатами симуляций (нижняя строка таблицы \ref{tab:randw_p_atm}).
Это показывает сильное влияние степенного шкалирования результатов вблизи нуля аргументов функций: чем меньше степенной коэффициент ($a$ и $s$ для $\pkn(N)$ и $\pkn(\Nun)$ соотвественно), тем больше абсолютная разница между пределом и ближайшим экспериментом (сильнее ''загиб'' функции у $1/N \to 0$, как видно на графиках \ref{fig:randw_p_full}).

\begin{table}[h]
\centering
\begin{tabular}{|c|c|c|}
\hline
 & a & s \\ \hline
$p^{(0)}$ & 0.202(7) & 0.25(1) \\ \hline
$p^{(1)}$ & 0.37(3) & 0.44(4) \\ \hline
$p^{(2)}$ & 0.272(6) & 0.323(7) \\ \hline
$p^{(3)}$ & 0.259(6) & 0.310(6) \\ \hline
\end{tabular}
\caption{Степенные коэффициенты шкалирующих функций вероятностей блужданий с фиксированной атмосферой (от 0 до 3, сверху вниз) от $N$ \eqref{eq:pi_N} (левый столбец), от $\Nun$ \eqref{eq:pi_Nu} (правый столбец).}
\label{tab:as_atm_compare}
\end{table}

При сравнении коэффициентов $a$ и $s$ функций \eqref{eq:pi_N} и \eqref{eq:pi_Nu} соответственно, проиллюстрированном в таблице \ref{tab:as_atm_compare}, численное равенство между функциями одинаковых величин почти не наблюдается.
Исключением является пара $a_1$ и $s_1$, касающиеся друг друга в пределах погрешностей, но в отсутствии закономерностей среди других величин, совпадение может быть случайно.
С другой стороны, аналогично сравнению степенных коэффициентов функций долей узлов \eqref{eq:pi_N} и \eqref{eq:pi_Nu} в разделе \ref{sec:n_final}, наблюдается соразмерность степенных коэффициентов ($a_1 > a_2 > a_3 > a_0$ и $s_1 > s_2 > s_3 > s_0$). 
Заметно также, что все степенные коэффициенты $s$ функций $\pkn(\Nun)$ больше аналогичных коэффициентов $a$ функций от $N$.
Получим, что число уникальных узлов $\Nun$ с точки зрения степенного коэффициента является более сильным аргументом для функций вероятностей $\pkn$, что логично, ведь атмосфера блуждания напрямую зависит от заполненности решетки посещёнными или уникальными узлами.

\begin{table}[h]
\centering
\begin{tabular}{|c|c|c|}
\hline
 & k & q \\ \hline
$p^{(0)}$ & -1.17(1) &  -1.142(9) \\ \hline
$p^{(1)}$ & 0.54(1) & 0.52(1) \\ \hline
$p^{(2)}$ & 0.596(4) & 0.585(5) \\ \hline
$p^{(3)}$ & 0.613(5) & 0.604(5) \\ \hline
\end{tabular}
\caption{Линейные коэффициенты шкалирующих функций вероятностей блужданий с фиксированной атмосферой (от 0 до 3, сверху вниз) от $N$ \eqref{eq:pi_N} (левый столбец) и от $\Nun$ \eqref{eq:pi_Nu} (правый столбец).}
\label{tab:kq_atm_compare}
\end{table}

Линейные коэффициенты $k, q$  функций \eqref{eq:pi_N} и \eqref{eq:pi_Nu} изображены для наглядности на таблице \ref{tab:kq_atm_compare} и так же содержат схожие друг другу закономерности.
Коэффициенты при $p^{(1)}$ чуть касаются друг друга в пределах погрешности, при $p^{(2)}$ линейные показатели близки, но чуть-чуть расходятся в пределах ошибок, в то время как $k_3$ и $q_3$ равны в пределах погрешности.
Пара коэффициентов $k_0$ и $q_0$ сильно отличаются по сравнению с ошибками, однако имеют одинаковый отрицательный знак.

В итоге мы наблюдаем ожидаемое равество пределов $p^{(i)}$ как функций от $N$ \eqref{eq:pi_N}, так и от $\Nun$ \eqref{eq:pi_Nu}. 
Остальные поведенческие показатели (линейные и степенные) отличаются, что говорит об отсутствии полной численной взаимозаменяемости таких величин, как число шагов блуждания $N$ и число уникальных узлов блуждания $\Nun$.
Несмотря на это, функции величин от обоих аргументов универсальны по пределу и в знаковом поведении.
Это позволило увидеть большую информативность числа уникальных узлов $\Nun$ как аргумента наблюдаемых величин.

\newpage

\subsection{Сравнение поведения атмосфер блужданий $\pkn$ и долей узлов $n_i$ модели RW}

Формула \eqref{eq:pkn}, описывающая атмосферу $a_t$ блуждания модели RW через число соседей конца блуждания $n_end$, указывает на возможное сходство свойств модели.
Очевидно, атмосфера характеризует локальное координационное число блуждания в его конечном узле. 
В данном разделе будет проведено сравнение ранее полученных результатов для долей узлов $n_i$ (в разделе \ref{sec:neigh}) и вероятностей атмосферы $\pkn$ (в разделе \ref{sec:atm}). 

Ранее для обоих свойств было подтверждено степенное шкалирование относительно как числа шагов $N$, так и среднего числа уникальных узлов блуждания $\Nun$.
На данный момент рассмотрим оценки коэффициентов шкалирующих функций от $\Nun$ (таблицы \ref{tab:n_i_u_log_log} для $n_i$ и \ref{tab:p_i_u_log_log} для $\pkn$).
Очевидно, что сходство необходимо искать между долей узлов с $v$ соседями и вероятностью $4-v$ атмосферы блуждания, где $v=\{1,2,3,4\}$.

\begin{table}[h]
\centering
\begin{tabular}{|c|c|c|c|}
\hline
v & $lim(n_v)$ & $lim(p^{(4-v)})$ \\ \hline
1 & 0.015(1) & 0.097(3) \\ \hline
2 & 0.053(2) & 0.141(3) \\ \hline
3 & 0.203(2) & 0.214(6) \\ \hline
4 & 0.741(5) & 0.59(2) \\ \hline
\end{tabular}
\caption{Пределы шкалирующих функций: долей узлов с $v$ соседями (центральный столбец) и вероятностей блужданий с фиксированной атмосферой $4-v$ (правый столбец) от $\Nun$ (столбцы $d$ в таблицах \ref{tab:n_i_u_log_log} для $n_i$ и \ref{tab:p_i_u_log_log} для атмосфер).}
\label{tab:n_vs_atm_d}
\end{table}

По таблице \ref{tab:n_vs_atm_d} видно, что несмотря на соразмерность пределов, в границах погрешностей они значительно отличаются. 

Можно предположить, что связь между ними существует - об этом говорит как подтверждённый степенной характер сходимости, так и схожесть знаков линейных коээфициентов - однако, она крайне слаба ввиду разной статистической мощности наблюдаемых величин - очевидно, что $\la n_v \ra$ охватывает геометрическое поведение всего блуждания, а $\la p^{(v)} \ra$ описывает поведение лишь на его концах, характер которых с увеличением длины блуждания становится некоррелируемым с поведением внутренних узлов.

\newpage

\subsection{Планируемая деятельность}

\begin{itemize}
\item 3-я итерация программного кода для симуляции модели Rand-Walk - добавление в модель аналога квадратной решётки с целью упрощения расчётов уникальных узлов и их соседей.
\end{itemize}
