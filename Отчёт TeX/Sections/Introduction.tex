\section{Введение}

\subsection{Одномерная модель Изинга}

Модель Изинга представляет собой решетку, в узлах которой расположены магнитные моменты, направленные "вверх"  или "вниз" , чему соответствует значение "cпина" на j-ом месте в решетке.

\[ \sigma_{j} = \pm1 \]

Энергией взаимодействия внешнего поля с моделью будем считать сумму взаимодействий поля \textit{h} с каждым из N моментов со спином $\sigma_{j}$

\begin{equation}\label{Hh}
  H_{h} = - \sum_{j = 1}^{N} h  \sigma_{j}   
\end{equation}

Внутренним взаимодействией между двумя соседними моментами считаем:

\begin{equation}\label{Hj}
  H_{J} = - \sum_{(i,j)} J  \sigma_{i}  \sigma_{j}
\end{equation}

Тогда Гамильтонианом системы из N спинов будет:

\begin{equation}\label{eq:Ham} 
   H = - h\sum_{j = 1}^{N}  \sigma_{j} - J \sum_{(i,j)} (\sigma_{1} \sigma_{2} + \sigma_{2} \sigma_{3} + \cdots + \sigma_{N-1} \sigma_{N})
\end{equation}

\subsection{Cтатсумма цепи Изинга общего случая ($h, J \neq 0 $) : периодич. гран. условия и Трансфер-матрица}

Для поиска решения данного случая воспользуемся методом \textbf{трансфер-матриц}.

Для начала перепишем формулу \eqref{eq:Ham} в другой форме:
\begin{equation}\label{eq:HamTrM}
    H = - \frac{h}{2}\sum_{j = 1}^{N}  (\sigma_{j} + \sigma_{j+1}) - J \sum_{j = 1}^{N} \sigma_{j}  \sigma_{j+1} 
\end{equation}

Учитывая периодические гран. условия ($\sigma_{N+1} = \sigma_{1}$), то формулы \eqref{eq:Ham} и \eqref{eq:HamTrM} тождественно равны.

Тогда статсумма такой модели будет равна:

\begin{equation}\label{ZTrM}
    Z = \sum_{\sigma} e^{-\beta H} = \sum_{\sigma} \prod_{j=1}^{N} \exp{(\bj\sigma_{j}\sigma_{j+1} + \frac{1}{2}\bh(\sigma_{j} + \sigma_{j+1}))} = \sum_{\sigma} \prod_{j=1}^{N} T(\sigma_{j}, \sigma_{j+1})
\end{equation}
    
Где $T(\sigma_{j}, \sigma_{j+1})$ - трансфер-матрица для двух соседних моментов. Поскольку один момент принимают лишь два значения ($\pm1$), а пара - уже четыре - (1,1),(1,-1),(-1,1),(-1,-1) - то, их матрица представляет с собой матрицу с элементами, соответствующими этим парам значений:
\begin{align}\label{TrM}
  T(\sigma_{j}, \sigma_{j+1}) = &
  \begin{pmatrix}
    \exp{(\bj+\bh)} & \exp(-\bj) \\
    \exp{(-\bj)} & \exp{(\bj-\bh)}
  \end{pmatrix}  
\end{align}

Если рассмотреть сумму произведений двух соседних матриц от j-1, j и j+1 внутри цепи при всевозможных значениях моментов, мы получим:

\[\sum_{\sigma = \pm 1} T(\sigma_{j-1}, \sigma_{j})T(\sigma_{j}, \sigma_{j+1}) = T^{2}(\sigma_{j-1}, \sigma_{j+1})\]

\subsection{Диагонализация Трансфер-матрицы}

Попробуем диагонализировать Трансфер-матрицу ($ T = R T^{D} R^{-1} $), тогда полное произведение матриц будет:

\[ \sum_{\sigma}\prod_{j}^{N} T(\sigma_{j}, \sigma_{j+1}) = R (T^{D})^{N} R^{-1} (\sigma_{1},\sigma_{N+1} = \sigma_{1})\]

Диагонализированная матрица будет выглядеть как:

\begin{align}
T^{D} = &
\begin{pmatrix}\label{Td}
  \lp & 0 \\
  0 & \lm
\end{pmatrix} \\ 
(T^{D})^{N} = &
\begin{pmatrix}
  \lp^{N} & 0 \\
  0 & \lm^{N}
\end{pmatrix}
\end{align}

Найдём собственные значения $\lpm$ и их собственные вектора:

\[ \lpm = e^{\bj}  \cosh{(\bh)} \pm Q\]
\[ Q = \sqrt{e^{2\bj} \cosh{(\bh)}^{2} - 2 \sinh{(2\bj)}} \]

\begin{align}\label{R}
R = &
\begin{pmatrix*}
  e^{\bj}\lpd & e^{\bj}\lmd \\
  1 & 1
\end{pmatrix*}\\
\label{RInv}
R^{-1} = &
\begin{pmatrix*}
  \frac{1}{2e^{\bj}Q} & 1-\frac{\lpd}{2Q} \\
  -\frac{1}{2e^{\bj}Q} & \frac{\lpd}{2Q}
\end{pmatrix*}
\end{align}

\[ \lpmd = e^{\bj}  \sinh{(\bh)} \pm Q\]

Эти формулы понадобятся нам позднее.

Поскольку нам нужен инвариантный след данной матрицы, т.к. матрица зависит от одного элемента, то достаточно $ Z = Tr(T^{D})^{N} $

Таким образом:

\begin{equation}\label{Zpbc}
Z\pbc = \lambda_{+}^{N} + \lambda _{-} ^{N}     
\end{equation}

\subsection{Cтатсумма цепи Изинга общего случая ($h, J \neq 0 $) : открытые гран. условия}

Расчёт статсуммы в данном случае сложнее, т.к. система не замкнута, и крайние значения не имеют внутреннего взаимодействия между с собой. Попробуем воспользоваться формулой \eqref{eq:HamTrM} с корректировкой под открытые условия:

\begin{equation}\label{eq:HamTrM2}
    H = - \frac{h}{2}\sum_{j = 1}^{N-1}  (\sigma_{j} + \sigma_{j+1}) - J \sum_{j = 1}^{N-1} \sigma_{j}  \sigma_{j+1} - \frac{h}{2}(\sigma_{1} + \sigma_{N})
\end{equation}

\begin{align*}\label{ZTrM}
    Z = \sum_{\sigma} e^{-\beta H} = \sum_{\sigma} \prod_{j=1}^{N-1} \exp{(\bj\sigma_{j}\sigma_{j+1} + \frac{1}{2}\bh(\sigma_{j} + \sigma_{j+1}))} exp(\frac{1}{2}\bh(\sigma_{1}+\sigma_{N})) = \\
    = \sum_{\sigma} \prod_{j=1}^{N-1} T(\sigma_{j}, \sigma_{j+1}) T^{'}(\sigma_{1}, \sigma_{N})
\end{align*}

Где $T'(\sigma_{1}, \sigma_{N})$ - трансфер-матрица для крайних моментов. От ранее расмотренных матриц она отличается отсутствием внутреннего взаимодействия, поэтому она представима в виде:
\begin{align*}
  T(\sigma_{1}, \sigma_{N}) = &
  \begin{pmatrix}
    \exp{(\bh)} & 1 \\
    1 & \exp{(-\bh)}
  \end{pmatrix}  
\end{align*}

К полному произведению применимы те же рассуждения, что и в периодическом случае: воспользовшись диагонализацией трансфер-матрицы T, мы получим:

\[ Z = \sum_{\sigma}\prod_{j}^{N-1} T(\sigma_{j}, \sigma_{j+1})T^{'}(\sigma_{1}, \sigma_{N}) = R (T^{D})^{N-1} R^{-1} T^{'}(\sigma_{1}, \sigma_{N})\]

Просуммировав элементы матрицы, полученной из данного произведения, мы получим:

\begin{equation}\label{Zobc}
    Z\obc = \lp^{N-1}(\frac{e^{\bj} \sinh{\bh}^2}{Q} + \frac{1}{e^{\bj}Q} +  \cosh{\bh}) - \lm^{N-1}(\frac{e^{\bj} \sinh{\bh}^2}{Q} + \frac{1}{e^{\bj}Q} -  \cosh{\bh}) 
\end{equation}

\subsection{Итоги}
Нам известна статсумма модели для общего случая: 
\[ Z\pbc = \lambda_{+}^{N} + \lambda _{-} ^{N} \]
 - для периодичного граничного условия
 
\[ Z\obc = \lp^{N-1}(\frac{e^{\bj} \sinh{\bh}^2}{Q} + \frac{1}{e^{\bj}Q} +  \cosh{\bh}) - \lm^{N-1}(\frac{e^{\bj} \sinh{\bh}^2}{Q} + \frac{1}{e^{\bj}Q} -  \cosh{\bh}) \]
- для открытого погран. случая, где 

\[ \lambda_{\pm} = e^{\bj}  \cosh{(\bh)} \pm Q\]
\[ Q = \sqrt{e^{2\bj} \cosh{(\bh)}^{2} - 2 \sinh{(2\bj)}} \]