\section{Introduction}

The model of self-avoiding walks is one of the extensively studied examples of linear polymers. 
Moreover, it is the simplest model to study critical behaviour while other models of polymer chains have different states in the thermal equilibrium in various solvent conditions. 
By adding interaction between nearest neighbors of monomers of the walk, we are enabled to study the phase transition fixed between solvent conditions, so the given polymer in the thermal equilibrium becomes extended in the good-solvent conditions and collapses in the poor-solvent one. 
This tricritical nature was described in \cite{Gennes1979}.

The impact of close-range interaction was studied precisely in models of magnetic polymers, where interaction between monomers became more complex after each monomer carried a spin and the strength of the nearest neighbors coupling became variable. 
This is the so-called Ising Model on a SAW conformation. 
In \cite{Garel1999}, the model was complicated by adding the external magnetic field and all conclusions about magnetic properties were made by comparing with the mean-field model. 
However, there are some geometric properties whose impact on magnetic properties is not clear while their studying may require more brute methods.

In the previous studies \cite{faizullina2021critical}, it was established that the Ising model on the self-avoiding walk conformations (SAWs) has a continuous type of phase transition. 
In this work, we continue to study geometric properties of this model and compare them with "parent" models and their modifications, such as the Ising model on the rectangular lattice, considered in \cite{Selke2006}, and two-dimensional interacting self-avoiding walks exactly in their respective critical regions. 
One of the suggestions is that models with similar geometric properties will also have the same magnetic properties, which can be observed while comparing values of Binder cumulants in the $\theta$-transition of models with the equal values of asphericities.

The rest of the paper is structured as follows. 
Section 2 summarizes the main related work and the previous results in this area. 
Section 3 describes the reseacrhed models, their main geometric and magnetic observables and the simulation methods. 
Sections 4 is divided in two parts, each on them presents the results of the two main directions of the work: the comparison the Ising-ISAW and the regular Ising models in critical areas and the research of the local coordination number in the Ising-ISAW model across several lattice modifications.
The conclusions and the future work are summarized in Section 5.
