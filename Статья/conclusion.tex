\section{Conclusion}

In the first part of the paper we researched critical geometrical properties of Ising-ISAW model on the square lattice, such as mean asphericity of the conformation and aspect ratio. The results were compared with the previous results of research of the Rectangular Ising model, in attempt to find universal magnetic properties in the critical area. The comparison of observed models showed complete dissimilarity of critical bevaviour of magnetic properties between them.

The behaviour of the new metric of compaction of confromation - proportion the monomers with fixed number of the nearest neighbors - was reseacrhed in the second part of the paper. The metric showed clear features of extended and swollen states of conformations between the $theta$-point. The metric also indicates the first-order phase transition for cubic model and the continuous type of transition on the triangle lattice, which clearly correlates this previous results in \cite{Foster2021, faizullina2021critical}

Found differencies between models showed that both models can be equally useful, so they cannot be interchangeable in random walk based algorithms with accent on local coordination number. In sociology both models can be applied as different strategies for analysis of connections in social networks.

Further work can be directed in research another case of local coordination number, such as atmosphere of SAW-conformation, defined as a number of free lattices nodes near the end of conformation.