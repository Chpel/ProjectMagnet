\section{Literature Review}
\label{sec:lit}
There are numerous studies of self-avoiding walk models with different interacting mechanisms. 
Some approaches have focused on critical behaviour of two-dimensional SAW models, by observing the geometrical properties \cite{Caracciolo2011, owczarek2000first, arkin2013gyration}. 
Researchers studied several shape factors, such as average asphericity and size ratios, which are essential in hydrodynamics, by using Monte-Carlo statistical methods. 
They have verified their computations by mean-field theory, widely used in low-dimensional models. 
As a result, they determined an elliptic average shape of the two-dimensional walks and calculated the average ratio of the two axes in the critical area.
Another area of growing interest is the critical behaviour of lattices with more available connections between single monomer, such as cubic and triangular Ising-ISAW \cite{Foster2021} and ISAW \cite{Tesi1996, Privman1986} models. 
A number of approaches presented outer-interacting models, designed for hydrodynamics processes \cite{LivneSAW1988, madras1988pivot}. 
Variety of researched lattices prompts studying still underresearched universality of the properties among different models and lattices, particularly geometric ones, which is researched in this study. 
One of the backbone directions of the work is the comparison of geometrical properties of Ising-ISAW model and ISAW model on several lattices, such as square, triangular and cubic ones.

With the significant development of technical characteristics of supercomputers, it became possible to simulate much longer conformations. 
It also paved the way for more accurate research of approximate behaviour of infinite-sized conformations, so it made an opportunity to research of the polymers of the real size. 
Some approaches have already estimated the scaling of the geometrical properties \cite{owczarek2008scaling}. 
Researchers have determined the linear scaling of atmosphere probability of non-interacting SAWs and by linear regression, have calculated the asymptotic limits. 
In this work, the particular attention is paid to analysis of the behaviour of more specific to the whole model properties. 
The researched geometrical properties of Ising-ISAW models are the local coordination number as a fraction of the monomers with fixed number of neighbours. 
The regression methods were required to estimate linear and non-linear scaling coefficients.
