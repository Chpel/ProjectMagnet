\section{Введение}

\setcounter{page}{1}

Модель блуждания без самопересечений является одной и наоболее часто изучаемой моделью линейного полимера. 
Решёточная структура окружающей среды модели позволяет не только определять способы перемещения блуждания в пространстве, но и исследовать модификации, дополненные внутренним взаимодействием - например, между ближайшими парами мономеров.
Базовым примером моделей с энергетической составляющей является взаимодействующее блуждание без самопересечений (далее - ISAW), чья энергия равна числу взаимодействий в системе.
Сама система варьируется константой силы взаимодействия между узлами, и тем самым, в условии термального равновесия, можно выделить два основных конформационных состояния - схлопнутый клубок и вытянутая глобула - между которыми расположена точка фазового перехода.
В работе \cite{Gennes1979} была доказана трикритичность модели.

Примером взаимодействия полимера со внешней средой можно назвать семейство адсорбирующих блужданий, 
вступающих в реакцию с некоторой поверхностью \cite{LivneSAW1988}.

Возможно усложнение внутреннего взаймодействия, путём внедрения спиновой подсистемы в конформацию 
с сохранением условия связи между ближайшими узлами.
Таким образом была получена модель Изинга на случайном блуждании без самоперечений (далее - IsingISAW).
Предшествующая ей регулярная модель Изинга так же варьируется константой силы взаимодействия, под действием которой система проявляет парамагнетические ими ферромагнетические свойства.
Часть исследований модели проводятся с использованием теории среднего поля - так были рассмотрены магнитные свойства модели IsingISAW с дополнительным внешним полем \cite{Garel1999}. 
Однако, существуют некоторые наблюдаемые величины модели, тесно связанные как с магнитными, так и с конформационными свойствами, чьё исследование требует более статистического подхода.
Так же важно учитывать многообразие решёточных структур: некоторые из них имеют слишком большую размерность для достижения аналитического решения, 
иные содержат внешне незначительные изменения по сравнению с ранее изученными аналогами, 
но в то же время их критические свойства оказываются полностью различны.

Основным способом исследования подобных моделей являются симуляции их подсистем алгоритмами Монте-Карло \cite{Worm, Wolff, madras1988pivot}.
Задачи отличаются периодами релаксации конформационной и спиновой подсистемы.
Условия симуляции одной из систем при фиксированном состоянии другой определяют задачи замороженного спинового или конформацинного беспорядка.
Задача размороженного беспорядка, в свою очередь, задаётся условием равного периода релаксации обоих подсистем, и является менее изученной.

В прошлой работе \cite{faizullina2021critical} было проведено исследование критического поведения модели IsingISAW на квадратной решётке.
Из основных результатов был определён непрерывный характер фазового перехода, а так же оценены критические показатели модели.
Подобное исследование проводилось и для трёхмерной модели \cite{foster2021critical}.
Также для квадратной решётки была рассмотрена новая геометрическая характеристика блуждания - доля узлов с фиксированным числом соседей. 
Одно из основных направлений данной выпускной квалификационной работы посвящено исследованию этой характеристики среди 
структурных модификаций модели IsingISAW на квадратных решётках при размерности d=2,3,4, а так же на треугольной 2D-решётке.

Ранее треугольная решётка была исследована в качестве модификации как взаимодействуего полимера ISAW \cite{Privman1986}, 
так и регулярной модели Изинга \cite{ShchurTriangle, selke2006critical}. 
В данной работе также исследуется критическое поведение модели IsingISAW на треугольной решётке, а также уточняются результаты для взаимодействующего полимера ISAW.
