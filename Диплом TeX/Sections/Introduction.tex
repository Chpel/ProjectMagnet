\section{Введение}

\setcounter{page}{1}

Линейный полимер - одна из классических моделей полимерной физики, 
с помощью которой исследуется взаимодействие молекулы вещества с разбавленным растворителем, 
или с другими молекулами, в случае концентрированного раствора \cite{Gennes1979}.
Линейный полимер представляется цепочкой мономеров, взаимодействующих как с раствором, так и друг с другом.
Каждый мономер содержит область исключенного объёма, отталкивающую другие, не связанные с ним полимером мономеры, 
тем самым не допуская нарушения линейной целостности цепочки.
Одной из математических интерпретаций полимера с исключенным объёмом вокруг его составляющих выступает случайное блуждание без самопересечений (self-avoiding walk, далее SAW) на некоторой решётке.
Конформацию полимера изображают как последовательность неповторяющихся узлов решётки, чем обеспечивается отсутствие самопересечений.
Последовательные пары блуждания соединены ребром решётки, что ограничивает слишком близкое размещение мономеров, запрещённое исключенным объёмом, а также задаёт участки вокруг узла блуждания, где могут лежать другие мономеры \cite{Gennes1979, Vanderzande1998}.

Между близко расположенными в пространстве мономерами действуют притягивающие
силы Ван-дер-Ваальса.
В то же время, полимер взаимодействует с молекулами растворителя:
"хорошим" для мономера считается растворитель, взаимодействие с которым считается энергетическим выгодным, нежели с ближайшими мономерами.
В таком случае полимер переходит в развернутое состояние, с малым числом близких связей между мономерами.
При взаимодействии с иным растворителем, ситуация обратна и полимер сворачивается в более плотную глобулу, увеличивая внутреннее взаимодействие.
Простейшей моделью, симулирующая пободное поведение полимера, является взаимодействующее блуждание без самопересечений на решётке
(далее - ISAW), чья энергия равна числу взаимодействий в системе. 
Свойства системы в термодиначеском равновесии меняются в зависимости от параметра, замещающего все взаимодействия системы константой взаимодействия между узлами.
Температура среды, обратно пропорциональная энергии цепочки, отображает свойства растворителя.

Таким образом, между двумя основными конформационными состояниями полимера, описанными выше, 
расположена точка фазового перехода математической модели ISAW, разделяющая состояния преимущества Ван-дер-Вальсовых сил, эффектов исключенного объёма или взаимодействия мономеров с растворителем.
В работе \cite{Gennes1979} была доказана трикритичность данной системы.

%Примером взаимодействия полимера со внешней средой можно назвать семейство адсорбирующих блужданий, 
%вступающих в реакцию с некоторой поверхностью \cite{LivneSAW1988}.

Существуют также полимеры с более сложным внутренним взаимодействием.
Так, магнитные полимеры обладают мономерами с магнитным моментом, взаимодействие между которыми
направлено как на притяжение,
так и на отталкивание близлежащих мономеров.
Система приобретает новые свойства, и теперь, в зависимости от вышеперечисленых ранее факторов, может проявлять парамагнетические свойства или наоборот, приобрести спонтанную намагниченность мономеров и, следовательно, ферромагнетические свойства.  
Аналогичные свойства добиваются в модели ISAW путём внедрения спиновой подсистемы в конформацию 
с сохранением условия связи между ближайшими узлами.
Таким образом была получена модель Изинга на случайном блуждании без самоперечений (далее - IsingISAW) \cite{Aerstens1992}.
Спиновая подсистема модели взята от регулярной модели Изинга на решётке, которая, под действием параметра константы взаимодействия, проявляет парамагнетические ими ферромагнетические свойства.

Основным способом исследования подобных моделей являются симуляции их подсистем алгоритмами Монте-Карло \cite{Worm, Wolff, madras1988pivot}.
Задачи отличаются временами релаксации конформационной и спиновой подсистемы.
Так, в задачах замороженного спинового или конформацинного беспорядка одна из подсистем имеет значительно большее время релаксации, чем другая.
Задача размороженного беспорядка, в свою очередь, задаётся условием равного времени релаксации обоих подсистем, и является менее изученной.

Часть исследований модели проводятся с использованием теории среднего поля - так были рассмотрены магнитные свойства модели IsingISAW с дополнительным внешним полем \cite{Garel1999}. 
Однако, существуют некоторые наблюдаемые величины, требующие более статистического подхода.
В отдельно взятой конформации полимера выделяются несколько внутренних структурных групп:
образованные в термодинамическом пределе кластеры мономеров, или блобы, содержат ядро кластера с минимальным взаимодействием с раствором, и граничный слой с частичным взаимодействией мономеров с раствором. 
Сами кластеры соединены мостами - одномерными цепочками из мономеров \cite{Gennes1979}.
Их отношение с ростом длины цепочки позволяет более тщательно исследовать структурные изменения полимера (такие как отношения поверхность-объём) в приближении к реальным моделям в виде цепочек бесконечно большой длины методом шкалирования конечной длины.

В контексте моделей блужданий на решётке, принадлежность мономера к той или иной структурной группе характеризуется его локальным координационным числом - числом связей, образованных с данным мономером, или числом ''соседей'' данного узла. 

В прошлой работе \cite{faizullina2021critical} было проведено исследование критического поведения модели IsingISAW на квадратной решётке: 
был определён непрерывный характер фазового перехода, а так же оценены критические показатели модели.
Подобное исследование проводилось и для трёхмерной модели в работе \cite{foster2021critical}.
Также для квадратной решётки была рассмотрена новая геометрическая характеристика блуждания - доля узлов с фиксированным числом соседей - и зафиксировано нетривиальное поведение отношения поверхность-объём с ростом константы взаимодействия J.
Одно из основных направлений данной выпускной квалификационной работы посвящено исследованию данной характеристики среди 
структурных модификаций модели IsingISAW на квадратных решётках при размерности d=2,3,4, а так же на треугольной 2D-решётке.
Так же будет исследовано влияние на поведение локального координационного числа эффектов взаимодействующих между мономерами сил, на примере невзаимодействующего СБС, и исключенного объёма, при рассмотрении простого случайного блуждания.

Ранее треугольная решётка была исследована в качестве модификации как взаимодействуего полимера ISAW \cite{Privman1986}, 
так и регулярной модели Изинга \cite{ShchurTriangle, selke2006critical}. 
В данной работе также исследуется критическое поведение модели IsingISAW на треугольной решётке, а также уточняются результаты прошлых исследований взаимодействующего полимера ISAW.

Дальнейшая работа устроена следующим образом:
в секции 2 в деталях описаны исследуемые модели, их модификации, наблюдаемые в рамках экспериментов величины, а также методы симуляций блужданий,
секция 3 посвящена исследованию локального координационного числа мономеров модели Ising-ISAW в контексте долей узлов с фиксированным числом соседей на различных решётках,
аналочиное исследование для простого случайного блуждания описано в секциях 4-6,
и, наконец, в секциях 7-8 исследуются критические свойства моделей взаимодействующих и магнитных полимеров на треугольной решётке.
