\section{Заключение}

В ходе первой части работы было исследовано поведение локального координационного числа модели Изинга на случайном блуждании без самоперечений, а так же в модели взаимодействующего блуждания.
Плотность связей была рассмотрена в модели на нескольких решётках в зависимости от константы J,
а так же в состоянии нулевого взаимодействия в зависимости от длины цепочки.
Методами Монте-Карло были определены характер зависимости средних долей узлов с фиксированным числом соседей от длины конформации и пределы долей на бесконечно большой длине цепочки, а так же проведено сравнение этих значений на предмет универсальности среди квадратной, треугольной, кубической и 4D-гиперкубической решёток.

Для проверки универсальности свойства средних долей узлов с фиксированным числом соседей между внутренними узлами и концами конформаций, было проведено сравнение полученных для Ising-ISAW на квадратной решётке пределов средних долей при бесконечно большой длине в точке J=0 с предельными значениями вероятностей атмосфер невзаимодействующего блуждания из работы \cite{owczarek2008scaling}. Результат показал явную разницу поведения внутреннего и граничного узла конформаций с ростом длины конформации.

Аналогичное исследование локального координацинного числа было проведено для простого случайного блуждания на квадратной решётке.
Результаты симуляций показали степенной характер шкалирования долей узлов с фиксированным числом соседей как от числа шагов случайного блуждания, 
так и от числа уникальных узлов блуждания.
Похожим свойством обладало и атмосфера простого случайного блуждания, несмотря на гораздо большее численное отличие шкалирующих показателей между величинами.

Другим важным направлением являлось исследование критических свойств моделей Ising-ISAW и ISAW на треугольных решётках:
были найдены и уточнены оценки точек фазового перехода моделей, а так же проверена применимость ранее найденных критических показателей для квадратных решёток в треугольных модификациях моделей. 