\section{Заключение}

В ходе первой части работы было исследовано поведение локального координационного числа модели Изинга на СБС, а так же в модели взаимодействующего блуждания.
Плотность связей была рассмотрена в модели на нескольких решётках в зависимости от константы J,
а так же в состоянии нулевого взаимодействия в зависимости от длины цепочки.
Методами Монте-Карло был определён линейный характер зависимости средних долей локального координационного числа в блуждании (далее, долей ЛКЧ) от длины конформации и пределы долей на бесконечно большой длине цепочки.
Так же было проведено сравнение этих значений на предмет универсальности среди квадратной, треугольной, кубической и 4D-гиперкубической решёток: 
асимптотические пределы долей оказались численно различны среди разных решёток, однако по знаковому поведению Ising-ISAW на треугольной решётке имеет больше общего с кубической решёткой, имеющей одинаковое с треугольной координационное число, чем с квадратной, имеющей равную треугольной размерность. 

Для сравнения поведения ЛКЧ между внутренним и граничным узлом конформации в пределе бесконечно большой длины цепочки,
было проведено сравнение полученных для невзаимодействующей модели Ising-ISAW пределов средних долей ЛКЧ с предельными значениями вероятностей атмосфер невзаимодействующего блуждания на квадратной решётке из работы \cite{owczarek2008scaling}. 
Результат показал небольшую, но значимую впределах погрешностей разницу между величинами в пределе бесконечно длиной цепочки, 
что говорит о значительной различии поведений ЛКЦ в узлах близких в концами блужданий по сравнению с внутренними узлами цепочки. 

Чтобы рассмотреть влияние исключенного объёма на поведение локального координационного числа, аналогичное исследование было проведено для простого случайного блуждания на квадратной решётке.
Результаты симуляций показали степенной характер шкалирования долей ЛКЧ как от числа шагов случайного блуждания, 
так и от числа уникальных узлов блуждания.
Похожим свойством обладало и атмосфера простого случайного блуждания, 
несмотря на гораздо большее численное отличие шкалирующих показателей между величинами, чем в невзамодействующем СБС.

Так же проводилось исследование критических свойств моделей взаимодействующих полимеров на треугольных решётках.
Точка фазового перехода модели TrIsing-ISAW значительно отличается от $\theta$-точки оригинальной модели на квадратной решётке. 
С другой, стороны методом коллапса данных была подтверждена применимость ранее найденных критических показателей $\nu$ и $\phi$ для квадратных решёток в треугольных модификациях моделей, что говорит об универсальности ряди критических свойств среди двумерных решёток.