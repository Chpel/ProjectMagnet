\documentclass[14pt]{article}

\usepackage{Packages}
\usepackage{Commands}
\graphicspath{ {C:/Users/Admin/Documents/Github/ProjectMagnet/Расчёты .ipynb/Images/} }

\begin{document}

\begin{titlepage}
\includepdf[pages={1}]{вкр_титул.pdf}
\includepdf[pages={1-2}]{тз_ВКР.pdf}
\end{titlepage}

\begin{abstract}
Данная работа посвящена исследованию магнитного полимера в виде модели случайного блуждания без самопересечений со спиновой подсистемой.
Модель исследуется на нескольких решётках: кубические решётки размерности $d=2,3,4$, а так же треугольная двумерная решётка.
Исследуется задача динамического беспорядка в каноническом амсамбле: спиновая подсистема и конформация модели имеют соразмерные времена релаксации, а длина цепочки фиксированная.
Доли узлов с фиксированным числом соседей (далее, доли ЛКЧ, локального координационного числа) определяются как мера локальной плотности блуждания.
С помощью методов Монте-Карло на основе алгоритма Червя, данные величины рассматриваются между моделями на решетках с равными размерностями или координационными числами.
Дополнительно исследуется шкалирование долей ЛКЧ в случае отсутствия в блуждании внутреннего взаимодействия, а так же без условия исключенного объёма.
Так же исследуются критические свойства моделей взаимодействующих и магнитных полимеров на треугольной решётке.
Результаты показали различие точек фазового перехода между моделями на треугольной и ранее известной квадратной решёток, однако их критические экспоненты оказались равны.
 \end{abstract}

\renewcommand\abstractname{Abstract}

\begin{abstract}
The study focuses on a model on a magnetic polymer as a chain under excluded volume effects with closely interacting Ising spins.
The model is studied on lattices with different numbers of dimensions $d$: cubic lattice with d=2,3,4 and triangular one with d=2. 
The study focuses on a problem of dynamic disorder in a canonical ensemble, as the model has comparable relaxation times between spin subsystem and conformation, and the fixed length of a chain.
The mean fraction of monomers on a chain with fixed number of neighbors (or bulks) is considered as a geometric measure of the local compaction of conformation.
Using Monte-Carlo methods in Worm-Algorithm, these measures are compared between chains with respective spin-to-spin coupling constant J in lattices having equal coordination number or dimensionality. 
The fractions of the LCN (local coordination number) are also researched by finite-size scaling methods in cases of zero-interaction and disabled excluded volume effects. 
The critical properties of the model were also researched on a triangular lattice.
Our results suggests that the "triangular" theta-point differs from the results for the square lattice, but models on both lattices are appeared to have equal critical expopents.
\end{abstract}
\newpage

\tableofcontents

%\if 0
\section{Введение}

\setcounter{page}{1}

Линейный полимер - одна из классических моделей полимерной физики, 
с помощью которой исследуется взаимодействие молекулы вещества с разбавленным растворителем, 
или с другими молекулами, в случае концентрированного раствора \cite{Gennes1979}.
Линейный полимер представляется цепочкой мономеров, взаимодействующих как с раствором, так и друг с другом.
Каждый мономер содержит область исключенного объёма, отталкивающую другие, не связанные с ним полимером мономеры, 
тем самым не допуская нарушения линейной целостности цепочки.
Одной из математических интерпретаций полимера с исключенным объёмом вокруг его составляющих выступает случайное блуждание без самопересечений (self-avoiding walk, далее SAW) на некоторой решётке.
Конформацию полимера изображают как последовательность неповторяющихся узлов решётки, чем обеспечивается отсутствие самопересечений.
Последовательные пары блуждания соединены ребром решётки, что ограничивает слишком близкое размещение мономеров, запрещённое исключенным объёмом, а также задаёт участки вокруг узла блуждания, где могут лежать другие мономеры \cite{Gennes1979, Vanderzande1998}.

Между близко расположенными в пространстве мономерами действуют притягивающие
силы Ван-дер-Ваальса.
В то же время, полимер взаимодействует с молекулами растворителя:
"хорошим" для мономера считается растворитель, взаимодействие с которым считается энергетическим выгодным, нежели с ближайшими мономерами.
В таком случае полимер переходит в развернутое состояние, с малым числом близких связей между мономерами.
При взаимодействии с иным растворителем, ситуация обратна и полимер сворачивается в более плотную глобулу, увеличивая внутреннее взаимодействие.
Простейшей моделью, симулирующая пободное поведение полимера, является взаимодействующее блуждание без самопересечений на решётке
(далее - ISAW), чья энергия равна числу взаимодействий в системе. 
Свойства системы в термодиначеском равновесии меняются в зависимости от параметра, замещающего все взаимодействия системы константой взаимодействия между узлами.
Температура среды, обратно пропорциональная энергии цепочки, отображает свойства растворителя.

Таким образом, между двумя основными конформационными состояниями полимера, описанными выше, 
расположена точка фазового перехода математической модели ISAW, разделяющая состояния преимущества Ван-дер-Вальсовых сил, эффектов исключенного объёма или взаимодействия мономеров с растворителем.
В работе \cite{Gennes1979} была доказана трикритичность данной системы.

%Примером взаимодействия полимера со внешней средой можно назвать семейство адсорбирующих блужданий, 
%вступающих в реакцию с некоторой поверхностью \cite{LivneSAW1988}.

Существуют также полимеры с более сложным внутренним взаимодействием.
Так, магнитные полимеры обладают мономерами с магнитным моментом, взаимодействие между которыми
направлено как на притяжение,
так и на отталкивание близлежащих мономеров.
Система приобретает новые свойства, и теперь, в зависимости от вышеперечисленых ранее факторов, может проявлять парамагнетические свойства или наоборот, приобрести спонтанную намагниченность мономеров и, следовательно, ферромагнетические свойства.  
Аналогичные свойства добиваются в модели ISAW путём внедрения спиновой подсистемы в конформацию 
с сохранением условия связи между ближайшими узлами.
Таким образом была получена модель Изинга на случайном блуждании без самоперечений (далее - IsingISAW) \cite{Aerstens1992}.
Спиновая подсистема модели взята от регулярной модели Изинга на решётке, которая, под действием параметра константы взаимодействия, проявляет парамагнетические ими ферромагнетические свойства.

Основным способом исследования подобных моделей являются симуляции их подсистем алгоритмами Монте-Карло \cite{Worm, Wolff, madras1988pivot}.
Задачи отличаются временами релаксации конформационной и спиновой подсистемы.
Так, в задачах замороженного спинового или конформацинного беспорядка одна из подсистем имеет значительно большее время релаксации, чем другая.
Задача размороженного беспорядка, в свою очередь, задаётся условием равного времени релаксации обоих подсистем, и является менее изученной.

Часть исследований модели проводятся с использованием теории среднего поля - так были рассмотрены магнитные свойства модели IsingISAW с дополнительным внешним полем \cite{Garel1999}. 
Однако, существуют некоторые наблюдаемые величины, требующие более статистического подхода.
В отдельно взятой конформации полимера выделяются несколько внутренних структурных групп:
образованные в термодинамическом пределе кластеры мономеров, или блобы, содержат ядро кластера с минимальным взаимодействием с раствором, и граничный слой с частичным взаимодействией мономеров с раствором. 
Сами кластеры соединены мостами - одномерными цепочками из мономеров \cite{Gennes1979}.
Их отношение с ростом длины цепочки позволяет более тщательно исследовать структурные изменения полимера (такие как отношения поверхность-объём) в приближении к реальным моделям в виде цепочек бесконечно большой длины методом шкалирования конечной длины.

В контексте моделей блужданий на решётке, принадлежность мономера к той или иной структурной группе характеризуется его локальным координационным числом - числом связей, образованных с данным мономером, или числом ''соседей'' данного узла. 

В прошлой работе \cite{faizullina2021critical} было проведено исследование критического поведения модели IsingISAW на квадратной решётке: 
был определён непрерывный характер фазового перехода, а так же оценены критические показатели модели.
Подобное исследование проводилось и для трёхмерной модели в работе \cite{foster2021critical}.
Также для квадратной решётки была рассмотрена новая геометрическая характеристика блуждания - доля узлов с фиксированным числом соседей - и зафиксировано нетривиальное поведение отношения поверхность-объём с ростом константы взаимодействия J.
Одно из основных направлений данной выпускной квалификационной работы посвящено исследованию данной характеристики среди 
структурных модификаций модели IsingISAW на квадратных решётках при размерности d=2,3,4, а так же на треугольной 2D-решётке.
Так же будет исследовано влияние на поведение локального координационного числа эффектов взаимодействующих между мономерами сил, на примере невзаимодействующего СБС, и исключенного объёма, при рассмотрении простого случайного блуждания.

Ранее треугольная решётка была исследована в качестве модификации как взаимодействуего полимера ISAW \cite{Privman1986}, 
так и регулярной модели Изинга \cite{ShchurTriangle, selke2006critical}. 
В данной работе также исследуется критическое поведение модели IsingISAW на треугольной решётке, а также уточняются результаты прошлых исследований взаимодействующего полимера ISAW.

Дальнейшая работа устроена следующим образом:
в секции 2 в деталях описаны исследуемые модели, их модификации, наблюдаемые в рамках экспериментов величины, а также методы симуляций блужданий,
секция 3 посвящена исследованию локального координационного числа мономеров модели Ising-ISAW в контексте долей узлов с фиксированным числом соседей на различных решётках,
аналочиное исследование для простого случайного блуждания описано в секциях 4-6,
и, наконец, в секциях 7-8 исследуются критические свойства моделей взаимодействующих и магнитных полимеров на треугольной решётке.

\section{Модели и методы}

В рамках данной работы определяется несколько моделей: 
в первую очередь определяется модель взаимодействующего блуждания без самопересечений ISAW. 
Энергия системы ISAW c конформацией $u$ (последовательности узлов решётки, на которых размещёна цепочка) 
фиксированной длины $N$ равна числу связей между ближайшими мономерами в цепочке \eqref{eq:ISAW_ham}:

\begin{equation}
\begin{array}{l}
\label{eq:ISAW_ham}
E(u) = J \sum_{\la i, j \ra} 1\ \ \ \ i,j \in u, |u|=N \\
Z = \sum_u \exp{(\frac{-E}{kT})}
\end{array}
\end{equation}

Модель рассматривается в каноническом амсамбле, поэтому статистическая сумма модели суммирует все возможные конформации $u$ длины $N$.

Так же определим модель Изинга на случайном блужданий без самопересечений (далее - Ising-ISAW).
В мономерах конформации длины $N$ встроена спиновая подсистема $\{s\}$, 
принимающая значение в узлах цепочки $+1$ или $-1$, вследствие чего энергия рассчитывается между ближайшими узлами цепочки как:

\begin{equation}
\label{eq:IsISAW_ham}
E(s,u) = J \sum_{\la i, j \ra} s_i s_j,\ \ \ \ i,j \in u, |u|=N
\end{equation}

Статическая сумма модели берётся по всем возможным последовательностям $\{s\}$ и конформациям $u$ фиксированной длины:

\begin{equation}
Z = \sum_s \sum_u \exp{(\frac{-E}{kT})}
\end{equation}

в обоих представленных моделях $T$ — температура, $k$ — постоянная Больцмана. 
Без потери общности можно считать $kT = 1$, тем самым оставляя J единственным самостоятелньым параметром модели.

Множество $\la i, j \ra$ под знаком суммирования обозначает пары узлов решётки, принадлежащие конформации модели $u$, между которыми лежит ребро исследуемой решётки.
В зависимости от выбранной решётки, для узла конформации меняется множество узлов решётки, 
которые могут считаться "ближайшими" к нему, ровно как и максимальное количество связей у одного мономера - так называемое "координационное число" решётки.
Так, квадратной решётке (левый рисунок \ref{fig:lattices}) соседями узла можно считать мономеры, расположенные сверху, снизу, слева и справа и него, 
в то время как в треугольной решётке соседними так же считаются и узлы на одной из диагоналей, проходящей через узел решётки,
а на кубической – к соседним приравнены узлы с теми же координатами на соседних плоскостях решётки.
Узел 4D-гиперкубической решётки имеет 8 соседей, каждый из которых отличается в одной координате на ±1 от рассматриваемого узла.

\begin{figure}
    \centering
    \includegraphics[width=0.3\textwidth]{SqLattice2.png}\ \ \ \ \ \ \ \ \ \ \ \ \ \ 
    \includegraphics[width=0.3\textwidth]{TriLattice2.png}
    \caption{Связи узлов в квадратной (слева) и треугольной решёток (справа). 
	Узлы пронумерованы последовательно (слева на направо и снизу вверх), в одном ряду решётки L узлов.}
    \label{fig:lattices}
\end{figure}

Первая часть выпускной квалификационной работы посвящена исследованию \textit{локального координационного числа} мономеров блуждания
в виде долей узлов блуждания с фиксированным числом соседей \eqref{eq:n_i}.
Минимальное исследуемое число соседей в моделях блужданий без самопересечений - два, что соответствует внутреннему узлу одномерной цепочки с соседями в виде предыдующего и следующего в последовательности.
Максимальное исследуемое число соседей соответствует координационному числу рассматриваемой решётки. 
Для каждого узла $u_i$ блуждания рассчитывается число его соседей $c_i$ \eqref{eq:c_i} и ведётся статистика узлов блуждания с таким же числом соседей.

\begin{equation}
\label{eq:c_i}
c_i = \sum_{\la u_i \textup{(fixed)}, j \ra} 1
\end{equation}

\begin{equation}
\label{eq:n_i}
n_k = \sum_{i=1}^{N-2}[c_i == k] / N
\end{equation}

Определим два основных критических вида перехода, происходящих в моделях - конформационный и магнитный.
Конформационный переход разделяет состояние рыхлой цепочки (при высоких температурах / низкой силе взаимодействия) и плотной глобулы (при низких температурах / сильном взаимодействии ближайших узлов).
Для исследования конформационного перехода исследуется расстояние между концами блуждания $R^2_N$:

 
\begin{equation}
\label{eq:R2_base}
	R^2_N = (u_{N-1} - u_{0})^2
\end{equation}

Состояния отличаются шкалированием радиуса между концами блужданий $\la R^2_N \ra$ относительно длины цепочки $N$:
при больших длинах цепочки (N >> 1) радиус \eqref{eq:R2_base} шкалируется по степенному закону.

\begin{equation}
	\la R^2_N \ra = N^{2\nu}(C + ...)
\end{equation}

В качестве основной магнитной характеристики модели рассматривается набор средних намагниченностей нескольких порядков:

\begin{equation}
\label{eq:IsISAW_m2}
	m^{k} = (\sum_{i \in u} \sigma_i / N)^k
\end{equation} 

Для поиска точки магнитного перехода рассматривается зависимость кумулянта Биндера $U_4$ \eqref{eq:IsISAW_U4} от константы J.

\begin{equation}
\label{eq:IsISAW_U4}
	U_4 = 1 - \frac{\la m^4 \ra}{3 (\la m^2 \ra)^2}
\end{equation}

Для регулярной модели Изинга было доказано \cite{Binder1981_Ising}, 
что с ростом размеров решётки значение кумулянта сходится к 0 при парамагнетическом состоянии, 
и к 2/3, что соответствует ферромагнетическим свойствам модели.
В работе \cite{Binder1981_Ising} так же было обнаружено нетривиальное значение кумулянта, почти независящее от размеров решётки.
Значение константы J, при котором достигалась нетривиальная сходимость, являлось точкой магнитного перехода модели.
Таким образом, предполагается, что точкой магнитного перехода модели является 
точка пересечения графиков кумулянта $U_4$ при разных длинах цепочки.

Ниже представлены известные точки фазового перехода моделей взаимодействующих блужданий (таблица \ref{tab:ISAW_T_c}), 
а так же моделей Изинга на регулярных решётках и СБС (таблица \ref{tab:Ising_T_c}).


\begin{table}[h]
    \centering
    \begin{tabular}{|c|c|c|}
        \hline
        Структура & Решётка & $J_{c}$ \\ \hline
        конформация СБС & Квадратная & $0.8340(5)$\cite{faizullina2021critical} \\ \hline
        конформация СБС & 3D-кубическая & $0.5263 \pm 0.055$\cite{foster2021critical}\\ \hline
        регулярная решётка & Прямоугольная & $\ln{(1 + \sqrt{2}) / 2}$\cite{Onsager}\\ \hline
    \end{tabular}
    \caption{Известные значения критической точки $J_c$ некоторых модицикаций модели Ising-ISAW и прямоугольного Изинга}
    \label{tab:Ising_T_c}
\end{table}

\begin{table}[h]
    \centering
    \begin{tabular}{|c|c|}
        \hline
        Решётка & $\beta_{c}$ \\ \hline
        Квадратная & $0.6673(5)$ \cite{caracciolo2011geometrical} \\ \hline
        3D-кубическая & $0.2779 \pm 0.0041$\cite{Tesi1996} \\ \hline
        Треугольная & $ 0.405 \pm 0.07$\cite{Privman1986} \\ \hline
    \end{tabular}
    \caption{Известные значения критической точки $\beta_c$ некоторых модицикаций модели ISAW}
    \label{tab:ISAW_T_c}
\end{table}

Для симуляции моделей в несколькими степенями свобод применяются методы Монте-Карло.
Исследуемая модель Ising-ISAW уже рассматривалась ранее в работах \cite{Garel1999, Papale2018} задаче замороженного беспорядка - когда свойства модели исследовались генерацией спиновой подсистемы на уже сгенерированных конформациях.
В нашей работе исследуется задача динамического беспорядка, в которой генерируются одновременно и блуждания фиксированной длины N, и спиновые состояния на ней.
Для генерации движущихся конформаций фиксированной длины используется алгоритм на основе метода Червя \cite{Worm}, 
в то время как генерация состояний спиновой подсистемы проводится с помощью кластерного алгоритма Вольфа \cite{Wolff}.
Полное описание используемого метода моделирования описаны в работе \cite{faizullina2021critical}.


\section{Геометрические свойства модели Ising-ISAW с точки зрения числа соседей в узлах}

\subsection{Введение}

В данном разделе мы изучаем такое геометрическое свойство модели, как доли узлов с фиксированным числов соседей. У каждого узла можно определить число соседей или количество близких связей на смежных ячейках исследуемой решётки (см. левый рисунок \ref{fig:lattices}). Рассмотрим пример конформации на квадратной решётке на рисунке \ref{fig:example_bulk}. Чёрные точки соответствуют узлам с 2-мя соседями, а последовательность таких узлов подряд в конформации можно интерпретировать как "одномерный" участок. Узлы с тремя соседями расположены, как правило, на границах кластеров, и отображены на примере синими треугольниками, в то время как узлы с четырьмя соседями (красные квадраты) типичны для узлов в глубине кластера.

\begin{wrapfigure}{r}{0.25\textwidth}
    \centering
    \includegraphics[width=0.24\textwidth, height=5cm]{update.png}
    \caption{Пример конформации на квадратной решётке с подсчётом соседей}
    \label{fig:example_bulk}
\end{wrapfigure}


Сначала, чтобы определить правильность алгоритма расчёта долей искомых узлов, были проведены симуляции Монте-Карло модели ISAW при J=0 на длинах N от 5 до 3600, а так же произведены расчёты вручную для цепочек малых длин - от 5 до 11. Результаты изображены на рисунке \ref{fig:ISAW_Bulk_J0} - разные типы расчётов полностью совпали, что говорит о правильности использумоего алгоритма.

\begin{figure}[]
    \centering
    \includegraphics[width=0.95\textwidth]{ISAWJ0_Bulk2-4.png}
    \caption{Зависимостей средних долей узлов конформации с фиксированным числом соседей (от 2 до 4) модели ISAW при J=0 от обратной длины 1/N при длинах конформации N=5-3600. 
	Пустые квадраты - результаты симуляций Монте-Карло, черные точки - расчёты полученные путём полного перебора возможных конформаций\cite{web:sawRepos}}
    \label{fig:ISAW_Bulk_J0}
\end{figure}

\subsection{Особенности ранних результатов на квадратной решётке}

Мы провели симуляции Монте-Карло для долей узлов с фиксированным числом соседей для моделей Ising-ISAW и ISAW с зависимостью от значения константы взаимодействия J для длин N=1000, 2500, 3600, 4900. Результаты изображены на рисунке \ref{fig:Ising_vs_ISAW__2D_bulk}, а также опубликованы в работе \cite{faizullina2021critical}. 

\begin{figure}[h!]
    \centering
    \includegraphics[width=0.95\textwidth]{bulk2-4_inset.png}
    \caption{Зависимость доли узлов конформации с двумя (слева), тремя (по центру) и четырьмя соседями (справа) у моделей Ising-ISAW (звезды) и ISAW на квадратной решётке от J. 
	Черной линией обозначена точка фазового перехода модели ISAW, красной - Ising-ISAW, на квадратной решётке (см. таблицу \ref{tab:crits}). 
	График взят из работы \cite{faizullina2021critical}}
    \label{fig:Ising_vs_ISAW__2D_bulk}
\end{figure}

На графиках \ref{fig:Ising_vs_ISAW__2D_bulk} примечательны значения в точке J=0 у графиков узлов с 2-мя (левый) и 3-мя (средний) соседями: было первоначальное предположение, что в пределе бесконечной длины конформации они будут равны 3/4 и 1/4 соответственно. Так же интересен вопрос универсальности данного свойства на других решётках: будут ли эти значения долей $n_{2}$ и $n_{3}$ при тех же условиях равны или хотя бы похожи в других решётках. 

\subsection{Сравнение модели Изинга и полимерной цепочки в решетках с 2-6 возможными соседями мономеров}

Рассмотрим средние доли узлов с фиксированным числом соседей в решётках, которые имеют от 2-х до 6-ти возможных соседей: в кубической, у которой 5-й и 6-й соседи мономера расположены в соседних плоскостях, и треугольной, где 5-й и 6-й сосед мономера лежат на диагонали, проходящей через данный узел (в данной решётке лишь одна плоскость, см. правый рисунок \ref{fig:lattices}).

\begin{figure}
    \centering
    \includegraphics[width=0.3\textwidth]{SqLattice2.png}\ \ \ \ \ \ \ \ \ \ \ \ \ \ 
    \includegraphics[width=0.3\textwidth]{TriLattice2.png}
    \caption{Связи узлов в квадратной (слева) и треугольной решёток (справа). 
	Узлы пронумерованы последовательно (слева на направо и снизу вверх), в одном ряду решётки L узлов.}
    \label{fig:lattices}
\end{figure}

График зависимости долей от константы взаимодействия J (используется в гамильтониане конформации по формуле \ref{eq:Ham}, однако в отличие от одномерного случая, где считаются связи между соседними по индексу узлами конформации, здесь считаются связи между узлами, лежащими на соседних ячейках исследуемой решётки) изображен на рисунке \ref{fig:Ising_vs_ISAW} - слева показаны результаты симуляций Монте-Карло на кубической решётке, справа - на треугольной решётке. Цвета графиков соответствуют длинам цепочек - N=100 зелёные, 300 синие, 600 красные и 1200 фиолетовые. Число шагов симуляций - от $10^{10}$ вдали от пиков до $10^{12}$ в районе пиков графиков. Вертикальными линиями отмечены точки критического перехода: 

\begin{table}[h!]
    \centering
    \begin{tabular}{|c|c|c|}
        \hline
        lattice & Ising-ISAW & ISAW \\ \hline
        square & 0.8340(5)\cite{faizullina2021critical} &  0.6673(5)\cite{caracciolo2011geometrical} \\ \hline
        triangular & Unknown & 0.41(7) \cite{Privman1986}\\ \hline
        cubic & $0.5263 \pm 0.055$\cite{foster2021critical} & $0.2779 \pm 0.0041$\cite{Tesi1996} \\ \hline
    \end{tabular}
    \caption{Значения J критических точек фазового перехода модели Изинга на случайном блуждании (Ising-ISAW) и гомополимера (ISAW) на квадратной, треугольной и кубической решётка соответственно (в порядке строк)}
    \label{tab:crits}
\end{table}

Результаты симуляций модели ISAW отмечены пустыми квадратами, а модели Ising-ISAW - звёздами. Примечательно, что графики зависимости долей от J данной модели значительно плавнее, чем у модели Изинга на случайном блуждании, а так же процессы уплотнения конформаций (когда доли $n_{2}$ и $n_{3}$ уменьшаются, а доли узлов с большим числом соседей увеличивается) начинаются раньше, пропорционально значению точки перехода $J_{c}$. Последнее, скорее всего, связано с тем, что точка перехода модели ISAW меньше, чем у Ising-ISAW (для кубической это известно, для треугольной просто предположение). Возможно, что при масштабировании левой части графиков кубической решетки относительно $J_{c}$ (то есть, от $0*J_{c}$ до $1*J_{c}$), мы бы получили примерно одинаковые графики.

В то же время, предельные значения у данных моделей совпадают - графики одинанаковых длин и решёток разных моделей исходят из одной точки при J=0 (что логично, ведь при J=0 поведение Ising-ISAW соответствует ISAW) и приходят в одну точку при J=1.

Данные модели Ising-ISAW в свою очередь отмечены на графике \ref{fig:Ising_vs_ISAW} звездочками. Стоит отметить, что при прохождении точки перехода в кубической решётке, графики долей узлов с любым числов соседей словно претерпевают скачок, усиливающийся с ростом длины цепочки, в отличие от треугольной решётки, где процесс непрерывен.

Говоря о свойствах Ising-ISAW кубической решётки, необходимо подчеркнуть, что в на графике $\la n_{3} \ra$ мы видим похожее поведение в J=0 - значение довольно близко к 0.25, стоит проверить предел значения доли узлов с 3-мя соседями в J=0 при бесконечной длине и характер приближения к нему, если таковой имеется. Значение $\la n_{2} \ra$ при J=0 визуально отличается от предполаемого $3/4$. В следующих разделах мы рассмотрим развитие значения долей $\la n_{2-6} \ra$ в точке J=0 (где модели ISAW и Ising-ISAW ведут себя идентично с обычным невзаимоидействующим блужданием SAW) на разных решётках на пределе бесконечной длины.

\begin{figure}
    \centering
    \includegraphics[width=0.95\textwidth, height=21.5cm]{Ising_vs_ISAW.png}
    \caption{Зависимость доли узлов моделей Ising-ISAW (звезды) и ISAW (квадраты) на кубической (слева) и треугольной решётках (справа) с 2-6 соседями (сверху вниз) от $J$ c длинами N = 100 (зеленые), 300 (красные), 600 (синие) и 1200 (фиолетовые). Вертикальные линии отмечают точки фазового перехода моделей \ref{tab:crits}}
    \label{fig:Ising_vs_ISAW}
\end{figure}

\newpage

\subsection{Алгоритм исследования характера зависимости значения долей узлов от длины при J=0}

Здесь рассматривается способ определения характера зависимости у графиков долей узлов с фиксированным числов соседей при J=0. Для примера взят случай $n_{2}$ у квадратной решётки модели Ising-ISAW. Первоначально рассматривается три возможных способа апроксимации результатов, варьирующихся зависимостью от обратной длины конформации $x = 1/N$:

\begin{enumerate}
    \item Линейная апроксимация 
    \begin{equation}
    \label{eq:linreg}
        y = a x + b
    \end{equation}
    \item Лог-линейная или экспоненциальная апроксимация 
    \begin{equation}
        y = b \exp{(a x)} + c 
    \end{equation}
    \item Степенная или лог-лог апроксимация
    \begin{equation}
        y = b x^{a} + c
    \end{equation}
\end{enumerate}

Чтобы гарантировано получить результат использовалась функция linregress из пакета scipy.stats, поэтому на данном этапе погрешностью результатов симуляций мы временно пренебрегаем. Так же, чтобы показать нагляднее характер апроксимации, графики соответсвующих способов фитирования будут рассмотрены в том же масштабе - линейный в линейном, экспоненциальный в лог-линейном, степенном в лог-лог-масштабе - таким образом графики фитов будут линейными. Результаты апроксимаций в порядке, изложенном в списке выше, изображены на рисунках \ref{fig:square_scale_full} и \ref{fig:square_scale_limited} - в левом столбце апроксимации записаны для данных цепочек с длинами от 100 до 4900, в правом - длины от 250 до 4900, чтобы оценить поведение модели на больших длинах, следовательно, ближе к нулю.

\begin{figure}[h!]

\begin{subfigure}{0.49\textwidth}
    \includegraphics[width=\textwidth]{square_n2_scaling.png}
    \caption{}
    \label{fig:square_scale_full}
\end{subfigure}
\hfill
\begin{subfigure}{0.49\textwidth}
    \includegraphics[width=\textwidth]{square_n2_scaling2.png}
    \caption{}
    \label{fig:square_scale_limited}
\end{subfigure}
\caption{Результаты апроксимации (оранжевая линия) данных Монте-Карло о долей узлов с двумя соседями $n_2$ модели Ising-ISAW на квадратной решётке (синие точки) различными способами на диапазоне длин 100-4900 (a) и 250-4900 (b)}
\end{figure}

Графики на рисунках \ref{fig:square_scale_full} и \ref{fig:square_scale_limited} показывают, что в данном случае экспоненциальная апроксимация ведёт себя как линейная (что логично вблизи нуля), поэтому можно рассматривать вместо первых двух только линейную. С другой стороны, степенная функция совсем не совпадает с графиком результатов. Более того, значение степени функции-фита настолько мало, что итоговая функция больше похожа на константную прямую.

Таким образом, в данном случае определён линейный характер зависимости. Теперь, чтобы оценить качество приближения при рассмотрении точек всё ближе и ближе к нулю, оценим ошибку фитирования - теперь мы можем использовать функцию curve-fit из пакета scipy.optimize.

\begin{table}[h]
    \centering
    \begin{tabular}{|c|c|c|} \hline
        N & a & b  \\ \hline
        100-4900 & -0.44(1) & 0.71292(4) \\ \hline
        250-4900 & -0.473(6) & 0.71299(2) \\ \hline
        400-4900 & -0.47(1) & 0.71298(2) \\ \hline
        1000-4900 & -0.48(6) & 0.71299(4) \\ \hline
    \end{tabular}
    \caption{Значения и погрешности коэффициентов линейного фитирования \eqref{eq:linreg} зависимости долей узлов с 2-мя соседями на квадратной решётке модели Ising-ISAW при J=0 от исследуемого интервала длин}
    \label{tab:a_b_n2_square}
\end{table}

Результаты использования других диапазонов точек на таблице \ref{tab:a_b_n2_square} показывают, что наиболее оптимальный фит (с наименьшей ошибкой) достигается при выборе точек от 250 до 4900. Это можно объяснить тем, что при выборе точек большего диапазона линейный характер будет выражен слабее, а при выборе точек меньшего диапазона количество рассматриваемых данных уменьшается, что приводит к росту ошибки (недостаточно статистики). Подобная операция была выполнена и для других чисел соседей и решёток (более подробные графики см. в Bulk2-6.ipynb в разделе "Расчёты .ipynb" \cite{web:ProjectMagnetRepos}), результаты представлены в следующем разделе в виде графиков для узлов с 2-мя и 3-мя соседями и в виде таблицы коэффициентов линейного фитирования \eqref{eq:linreg} без графиков. 

Результаты линейного фитирования при выборе разной наименьшей рассматриваемой длины можно увидеть на таблицах \ref{tab:n24_fit_coeff_100} и \ref{tab:n24_fit_coeff_200}. По погрешностях на первых строках обеих таблиц понятно, что оптимальным диапазоном будет 250-4900. Для 3-4D-гиперкубических решёток так же заметно (по погрешностям соответствующих строк), что отбрасывание длины N=100 из рассмартиваемых улучшило точность результатов. Единственное исключение - треугольная решётка: на ней линейный характер результатов настолько заметен, что при отбрасывании наименьшей длины N=100 ошибка увеличивается (недостаточность статистики стала сильнее, а ''линейность'' не изменилась).

\newpage

\subsection{Сравнение геометрических свойств модели Изинга на треугольной решётке с квадратной в J=0}

\begin{figure}
    \centering
    \includegraphics[width=0.95\textwidth]{n23_linear.png}
    \caption{Зависимость средней доли узлов с 2-мя соседями (слева) и 3-мя (справа) от обратной длины 1/N в модели Изинга на случайном блуждании на квадратной, треугольной, кубической и гиперкубической (сверху вниз). Синие точки описывают результаты симуляций Монте-Карло, оранжевая линия - график линейной апроксимации результатов, ошибки рассчитаны с учётом погрешностей полученных данных. Коэфициенты и диапазоны длин рассматриваемых данных записаны в таблице \ref{tab:n24_fit_coeff}}
    \label{fig:all_n23_bulk}
\end{figure}

На графике \ref{fig:all_n23_bulk} наглядно показано сравнение приближений долей "одномерных" участков (то есть, долей мономеров с двумя соседями) и узлов с тремя соседями в цепочках на квадратной, треугольной, кубической и гиперкубической решётках. Для расчётов долей на треугольной решётке были использованы длины 100-1200, для квадратной - 250-4900, для кубической и гиперкубической - 200-1200. Фитирование долей треугольной решётки имеет отчётливый линейный характер, даже в приближении на короткие длины. Линейность долей прямоугольных решёток всех размерностей также подтверждается (с учётом погрешности расчётов с наибольшей длиной). 

\begin{table}[h]
    \centering
    \begin{tabular}{|c|c|c|c|c|c|c|} \hline
         & \multicolumn{2}{|c|}{$\la n_{2} \ra$} & \multicolumn{2}{|c|}{$\la n_{3} \ra$} & \multicolumn{2}{|c|}{$\la n_{4} \ra$}\\ \hline
         Lattice & a & b & a & b  & a & b  \\ \hline
        Square & -0.44(1) & 0.71291(4) & -0.843(8) & 0.25297(3) &  -0.154(3) & 0.03412(1)  \\ \hline
        Triangular & 0.492(2) & 0.35989(1) & -0.519(3) & 0.37410(1) & -0.609(4) & 0.19080(1)  \\ \hline
        Cubic & 0.37(2) & 0.67440(7) &  -1.24(1) & 0.26005(5) & -0.525(5) & 0.05758(1) \\ \hline
        Hypercubic & 0.15(2) & 0.71978(9) & -1.20(1) & 0.22080(6) & -0.468(5) & 0.04589(2)\\ \hline
    \end{tabular}
    \caption{Коэффициенты прямых, полученные линейным фитированием \eqref{eq:linreg} данных симуляций Монтекарло по долям улов с 2-4 соседями из рисунков \ref{fig:all_n23_bulk} для длин N от 100 до 4900 (для квадратной) и 1200 (для остальных решёток)}
    \label{tab:n24_fit_coeff_100}
\end{table}

\begin{table}[h]
    \centering
    \begin{tabular}{|c|c|c|c|c|c|c|} \hline
         & \multicolumn{2}{|c|}{$\la n_{2} \ra$} & \multicolumn{2}{|c|}{$\la n_{3} \ra$} & \multicolumn{2}{|c|}{$\la n_{4} \ra$}\\ \hline
         Lattice & a & b & a & b  & a & b  \\ \hline
        Square & -0.473(6) & 0.71299(1) & -0.809(3) & 0.25291(1) &  -0.145(4) & 0.03410(1)  \\ \hline
        Triangular & 0.491(3) & 0.35989(1) & -0.523(6) & 0.37411(1) & -0.603(8) & 0.19079(2)  \\ \hline
        Cubic & 0.418(1) & 0.67429(3) &  -1.27(1) & 0.26012(2) & -0.538(4) & 0.05761(1) \\ \hline
        Hypercubic & 0.26(1) & 0.71958(3) & -1.27(1) & 0.22720(2) & -0.494(6) & 0.04596(1)\\ \hline
    \end{tabular}
    \caption{Коэффициенты прямых, полученные линейным фитированием \eqref{eq:linreg} данных симуляций Монтекарло по долям улов с 2-4 соседями из рисунков \ref{fig:all_n23_bulk} для длин N от 250 до 4900 (для квадратной) и от 200 до 1200 (для остальных решёток)}
    \label{tab:n24_fit_coeff_200}
\end{table}

\begin{table}[h]
    \centering
    \begin{tabular}{|c|c|c|c|c|c|c|c|c|c|} \hline
         & \multicolumn{3}{|c|}{$\la n_{2} \ra$} & \multicolumn{3}{|c|}{$\la n_{3} \ra$} & \multicolumn{3}{|c|}{$\la n_{4} \ra$}\\ \hline
         Lattice & a & b & N & a & b & N & a & b & N \\ \hline
        Square & -0.473(6) & 0.71299(2) & 250-4900 & -0.809(4) & 0.25291(1) & 250-4900 & -0.145(4) & 0.03410(1) & 250-4900  \\ \hline
        Triangular & 0.492(2) & 0.35989(1) & 100-1200 & -0.519(3) & 0.37410(1) & 100-1200 & -0.609(4) & 0.19080(1) & 100-1200 \\ \hline
        Cubic & 0.42(1) & 0.67429(3) & 200-1200 & -1.270(7) & 0.26012(2) & 200-1200 & -0.538(4) & 0.05671(1) & 200-1200 \\ \hline
        Hypercubic & 0.26(1) & 0.71957(3) & 200-1200 & -1.27(1) & 0.22721(2) & 200-1200 & -0.494(6) & 0.04595(1) & 200-1200\\ \hline
    \end{tabular}
    \caption{Коэффициенты прямых, полученные линейным фитированием \eqref{eq:linreg} данных симуляций Монтекарло по долям улов с 2-4 соседями из рисунков \ref{fig:all_n23_bulk} - наилучшие приближения с подбором диапазона длин для каждого графика (в столбце N)}
    \label{tab:n24_fit_coeff}
\end{table}

Из таблицы \ref{tab:n24_fit_coeff} по первым двум строках, отображающим данные о прямых-фитов квадратной и треугольной решётки соответствено, сходства между одномерием треугольной и квадратной решётки с точки зрения коэфициентов фитирования $a$ и $b$ \eqref{eq:linreg} почти не наблюдается - они имеют как разные значения свободных членов, так и значения и даже (в случае 2-х соседей) знаки коэффициента наклона, разница который значительно превышает погрешность фита. 

Значение свободного члена $b$ для $\la n_{2} \ra$, то есть предела значения долей при бесконечной длине цепочки, у квадратной и треугольной решётки (первый блок первых двух строк таблицы \ref{tab:n24_fit_coeff}) отличается почти в два раза: $0.71299(2)$ и $0.35989(1)$ (что логично, ведь в треугольной решётке диагональные ячейки так же считаются соседними, поэтому половина поворотов конформации добавит соседей).



\subsection{Сравнение геометрических свойств модели Изинга на решётках с большим числом соседей в J=0}

\begin{table}[]
    \centering
    \begin{tabular}{|c|c|c|c|c|c|c|} \hline
         & \multicolumn{3}{|c|}{$\la n_{5} \ra$} & \multicolumn{3}{|c|}{$\la n_{6} \ra$}  \\ \hline
        Lattice & a & b & N & a & b & N \\ \hline 
        Triangular & -0.274(2) & 0.063145(6) & 100-1200 & -0.055(1) & 0.012081(2) & 100-1200\\ \hline
        Cubic & -0.100(2) & 0.007536(4) & 200-1200 & -0.0074(2) & 0.000452(1) & 200-1200 \\ \hline
        Hypercubic & -0.102(2) & 0.00658(1) & 200-1200 & -0.0140(3) & 0.000659(1) & 200-1200\\ \hline
    \end{tabular}
    \caption{Коэффициенты прямых, полученные линейным фитированием \eqref{eq:linreg} данных симуляций Монтекарло по долям улов с 5-6 соседями}
    \label{tab:n56_fit_coeff}
\end{table}

\begin{table}[]
    \centering
    \begin{tabular}{|c|c|c|c|c|c|c|} \hline
         & \multicolumn{3}{|c|}{$\la n_{7} \ra$} & \multicolumn{3}{|c|}{$\la n_{8} \ra$}  \\ \hline
        Lattice & a & b & N & a & b & N \\ \hline 
        Hypercubic & -0.0011(1) & 0.0000420(3) & 200-1200 & -0.000024(35) & 0.0000010(1) & 200-1200\\ \hline
    \end{tabular}
    \caption{Коэффициенты прямых, полученные линейным фитированием \eqref{eq:linreg} данных симуляций Монтекарло по долям улов с 7-8 соседями}
    \label{tab:n78_fit_coeff}
\end{table}

Здесь мы сравниваем линейное фитирование результатов симуляций Монте-Карло треугольной решётки с кубической, имеющей такое же количество возможных соседей, а так же результаты для гиперкубической решётки в J=0. Коэффициенты линейного фитирования \eqref{eq:linreg} отображены в таблицах \ref{tab:n24_fit_coeff} и \ref{tab:n56_fit_coeff}: поскольку в таких условиях плотность конформаций минимальна, доля узлов с 7 и 8 соседей в конформациях на гиперкубической решётке почти нулевая, что видно по таблице \ref{tab:n78_fit_coeff}, поэтому мы рассматриваем число соседей лишь от 2 до 6. 

Рассматривая средние строки таблицы \ref{tab:n24_fit_coeff}, где записаны коэффициенты прямых фитирования для $n_{2}$ и $n_{3}$ треугольной и кубической решётки соответственно, а так же средние графики на рисунке \ref{fig:all_n23_bulk}, мы видим примерно ту же ситуацию как и в случае сравнения треугольной с квадратной - кубическая решётка на графике \ref{fig:all_n23_bulk} показывает почти чёткий линейный характер приближения в пределах погрешности наибольших длин (для n=3 линейно видна значительно лучше), но ни коэффициенты наклона $a$, ни значения свободных членов $b$ не имеют никакого сходства. Единственное отличие от сравнения с квадратной решёткой - графики соответствующих долей треугольной, кубической и гиперкубической решёток имеют одинаковое поведение с точки зрения знака наклона, что действительно и для долей узлов с больший числом соседей. Можно утверждать, что треугольная решётка с точки зрения поведения доли одномерных участок больше похожа на кубическую решётку, нежели квадратную, однако точной численной универсальности (например, почти равных в пределах погрешности коэффициентов) поведения доли "одномерных" участков между ними при бесконечно больших длинах конформации не обнаружена.

Единственная пара коэффициентов, которая оказалась равна в пределах погрешности, являются коэффициенты наклона у линейного фитирования $a$ \eqref{eq:linreg} для долей узлов с 3-мя соседями $\la n_{3} \ra$ у кубической и гиперкубической решёток (см. таблицу \ref{tab:n24_fit_coeff}).


\subsection{Число соседей и атмосферы блужданий}
\label{sec:Prellberg}

В статье \cite{owczarek2008scaling} в пространстве невзаимодействующих случайных блужданий без самопересечений было рассмотрено так свойство конформации, как "атмосфера" - количество возможных направлений для удлинения цепочки длины N или количество возможных N+1-х узлов.

Мы предполагаем, что данное свойство имеет связь с числом соседей при рассмотрении процесса удлинения цепочки и такие величины, как доля узлов цепочки $\la n_{i} \ra$ с фиксированным числом соседей и вероятность конформации иметь атмосферу $k$ - $p^{(k)}$ - по-разному описывают одно и то же поведение цепочек с точки зрения их плотности.

\begin{figure}
    \centering
    \includegraphics[width=0.5\textwidth]{Atmos_to_neibors_p3.png}
    \includegraphics[width=0.5\textwidth]{Atmos_to_neibors_p2.png}
    \includegraphics[width=0.5\textwidth]{Atmos_to_neibors_p1.png}
    \caption{Пример удлинения цепочки на квадратной решётке с атмосферой 3,2,1 (сверху вниз): слева изображена конформация до удлинения, справа - после, возможные способы добавить новый узел отмечены оранжевым, разметка узлов по количеству соседей соответствует рисунку \ref{fig:example_bulk}}
    \label{fig:atmos_neighs}
\end{figure}

Рассмотрим верхний рисунок \ref{fig:atmos_neighs}: если конец цепочки длины N (назовём его "N-ым узлом") имеет атмосферу три (три оранжевые точки вокруг правого конца), то при добавлении нового N+1-го узла N-й будет иметь два соседа: N-1-й и N+1-й узлы (бывший правый конец стал черной точкой). 

Так же при атмосфере 2 - как на среднем рисунке \ref{fig:atmos_neighs} - когда, уже имея два соседа (черная конечная точка) и две возможности для удлинения, N-ый узел при удлинении будет иметь 3 соседа (треугольник в том же месте на правой половине). 

И наконец, при атмосфере 1 (последний рисунок \ref{fig:atmos_neighs}) удлинение цепочки единственным возможным способом (одна оранжевая точка) приведёт к тому, что старый конец цепочки будет иметь 4 соседа (красный квадрат вместо треугольника). Примеры таких явлений можно увидеть на рисунке \ref{fig:atmos_neighs}. Очевидно, что случай удлинения при атмосфере 0 рассмореть невозможно, и провести аналогию с соседями нельзя.

Подобная интерпретация данных свойств в контексте удлинения цепочки показывает, что событие "цепочка длины N имеет атмосферу 3/2/1" при удлинении однозначно перехоходит к состоянию "N-й узел цепочки (теперь предпоследний) имеет 2/3/4 соседа" соответственно.

С другой стороны, подобная интерпретация атмосферы Преллберга не учитывает перерасчёт соседей у других узлов после удлинения цепочки - так, на примере атмосферы 1 (на нижнем рисунке \ref{fig:atmos_neighs}) видно, что у одного из узлов, кроме конечного (бывшая черная точка справа), так же увеличилось число соседей (с 2-х до 3-х), тем самым она стала поверхностным узлом (синим треугольником в том же месте на правой половине).

Проведём сравнение долей узлов в фикс. числом соседей в модели Ising-ISAW при J=0 и вероятность конформации модели невзаимодействующего блуждания иметь атмосферу k в пределе на бесконечно большую длину на квадратной решётке. 

\begin{table}[h]
    \centering
    \begin{tabular}{|c|c|c|c|}
    \hline
    k & $p^{(k)}$ & i & $b(\la n_{i} \ra)$ \\ \hline
    3 & 0.711 14(3) & 2 & $0.71299(2)$ \\ \hline
    2 & 0.225 00(2) & 3 & $0.25291(1)$ \\ \hline
    1 & 0.054 76(1) & 4 & $0.03410(1)$\\ \hline
    0 & 0.009 096(4) & - & - \\ \hline
    \end{tabular}
    \caption{Сравнение свободных членов линейных приближений вероятностей у конформации иметь n-ю атмосферу (слева) и долей мономеров с i соседями (справа) в зависимости от обратной длины конформации 1/N}
    \label{tab:Prellb_Compare}
\end{table}

На таблице \ref{tab:Prellb_Compare} слева изображены значения свободных членов графика зависимости вероятности гомополимерной цепочки иметь атмосферу k в статье \cite{owczarek2008scaling}, то есть вероятность, что второй конец цепочки бесконечно большой длины N имеет k возможных направления для удлинения и следовательно, k возможных узлов, которые могут стать новым узлом в цепочке. Справа изображены значения свободных членов приближений графиков долей узлов с i соседями. Хотя все значения отличаются больше чем на погрешность расчётов, однако нельзя не заметить довольно близкое сходство $p^{(3)}$ и свободного члена $\la n_{2} \ra$, хотя сами приближения имеют противоположные по знаку наклоны. 

В частной переписке с автором статьи была предложена следующая коррекция результатов \cite{web:PrellbergPrivate}: поскольку мы рассматриваем состояние при котором удлинения точно произойдёт, то сравнивать необходимо именно условные вероятности вида P({цепочка имеет атмосферу k | удлинение возможно}) = P({цепочка имеет атмосферу k}) / P({цепочка имеет положительную атмосферу}):

\begin{equation*}
    p^{(1/2/3)'} = p^{(1/2/3)} / (p^{(1)} + p^{(2)} + p^{(3)})
\end{equation*}

Рассмотрим такую ''приведённую'' вероятность атмосфер и сравним с результатами для долей соседей.

\begin{table}[h]
    \centering
    \begin{tabular}{|c|c|c|c|}
    \hline
    k & $p^{(k)'}$ & i & $b(\la n_{i} \ra)$ \\ \hline
    3 & 0.7177 & 2 & $0.71299(2)$ \\ \hline
    2 & 0.2271 & 3 & $0.25291(1)$ \\ \hline
    1 & 0.0553 & 4 & $0.03410(1)$\\ \hline
    \end{tabular}
    \caption{Вероятности у конформации иметь k-ю атмосферу (слева) и долей мономеров с i соседями (справа) в пределе бесконечной длины в случае гарантированно возможного удлинения}
    \label{tab:Prellb_Compare2}
\end{table}

Разница между $p^{(3)'}$ и $(\la n_{2} \ra)$ увеличилась. Остальные величины так же не удалось приравнять в пределах погрешности, что говорит о том, что величины обозначают несколько разные поведения модели.




\include{Sections/DS_Bulk}
\subsection{Зависимость доли уникальных узлов от количества шагов}

В данном подразделе проверяется численная эквивалентность фитирующих функций долей узлов $n_1-n_4$: $f_i$ \eqref{eq:n_i_log_log}, имееющей прямую зависимость от числа шагов $N$ и $g_i(\Nun)$ \eqref{eq:n_i_u_log_log}, со сложной зависимостью от $N$.

\begin{equation}
	\la n_i \ra = g_i(\Nun) = q_i (1/\Nun)^{s_i} + d_i
	\label{eq:gi_approx1}
\end{equation}

Из результатов прошлого подраздела были получены коэффициенты фитирующей функции $\nun$ (строка $\nun$ таблицы \ref{tab:n_i_log_log}). Определим их как $k_u, a_u, b_u$ соответственно и раскроем их в функции аргумента: 

\begin{equation}
	\Nun = N \nun(N) = N (k_u (1/N)^{a_u} + b_u)
	\label{eq:gi_appprox2}
\end{equation}

Рассмотрим аппроксимацию произвольной функции $g_i(\Nun)$ с выражением её аргумента через $N$.
Подставим \eqref{eq:gi_appprox2} в \eqref{eq:gi_approx1} и проведём линеаризацию - сначала $1/\Nun(N)$ при $1/N \to 0$, а затем $(1/\Nun(N))^{s_i}$ при $1/N \to 0$:

\begin{Large}
\begin{equation*}
\begin{array}{l}
1)\ \ \ \ \ (N (b_u + k_u(1/N)^{a_i})^{-1} = ( N b_u)^{-1} (1 + \frac{k_u}{b_u} \frac{1}{N^{a_u}})^{-1} = \frac{1}{b_u N} (1 - \frac{k_u}{b_u} \frac{1}{N^{a_u}} + O(\frac{1}{N^{2a_u}})) \\
\\
2)\ \ \ \ \ ( - // - )^{s_i}  = \frac{1}{(b_u N)^{s_i}} \ \  (1 - \frac{k_u}{b_u} \frac{1}{N^{a_u}} + O(\frac{1}{N^{2a_u}}))^{s_i} = \frac{1}{(b_u N)^{s_i}}\ \ (1 - \frac{k_u s_i}{b_u} \frac{1}{N^{a_u}} + O(\frac{1}{N^{2a_u}}))
\end{array}
\end{equation*}
\end{Large}

Итоговое выражение примет следующий вид:

\begin{large}
\begin{equation}
g_i(N) = \frac{k_i}{b_u^{s_i}} \frac{1}{N^{s_i}} - \frac{k_i s_i k_u}{b_u^{a_i+1}} \frac{1}{N^{a_u+s_i}} + d_i + O(\frac{1}{N^{2 a_u +s_i}}),\ \ \ \ \ N \to \infty
\end{equation}
\label{eq:g_n_expect}
\end{large}

Таким образом, мы свели функцию $g_i(\Nun)$ \eqref{eq:gi_approx1} к функции вида \eqref{eq:n_i_log_log}, сохранив дополнительные степенные поправки. 
Очевидно, линеризация повлияет на поведение в функции области небольших длин блуждания, поэтому оценивать теорически ожидаемые линейный и степенной коэффициент по полученной функции \eqref{eq:g_n_expect} невозможно.
Это объясняет различие коэффициетов $k_i, s_i$. 

С другой стороны, проведенные преобразования не дали никакой поправки для асимптотического предела $d_i$ - следовательно, вне зависимости от взятого аргумента, $N$ или $\Nun$, функции соответствующих долей узлов с фиксированным числом соседей $f_i$ и $g_i$ должны сходиться на бесконечности в одной точке, а столбцы $b$ и $d$ таблиц \ref{tab:n_i_log_log} \ref{tab:n_i_u_log_log} соответственно - равными в пределах погрешностей.
Это так же подтверждается тем, что если $N \to \infty$, то и, очевидно $\Nun \to \infty$, поскольку $b_u > 0$.  

Рассмотрим графики трёх функций на каждую долю $n_1$-$n_4$: как функцию $f_i(N)$, как функцию $g_i(N\nun(N))$, а так же аппроксимацию второй функции \eqref{eq:g_n_expect}.

Графики функций в линейном масштабе изображены на рисунке \ref{fig:ni_fn_vs_gNun}. По ним видно, что функция $f(N)$ и $g(\Nun(N))$ почти не имеют отличий, что говорит о полном взаимозаменяемости аргументов и правильности полученных результатов на небольших длинах. Зелёная линия соответствует аппроксимирующему виду $g(\Nun(N))$ и имеет поправку, уменьшающуюся при стремлении $N$ к бесконечности, но так же визуально сливается с первыми двумя функциями в области больших $N$.

\begin{figure}
\centering
\includegraphics[width=\textwidth]{n_i_fN_vs_gNun.png}
\caption{Доли узлов $n_1-n_4$ в линейном масштабе как функции от количества шагов (синий пунктир) \eqref{eq:n_i_log_log}, а так же сложные функции от количества уникальных узлов от количества шагов (оранжевая линия -- прямая подстановка функций  \eqref{eq:gi_appprox2} в \eqref{eq:gi_approx1}, зелёная - аппроксимация \eqref{eq:g_n_expect}), по горизонтали -- обратное количество шагов блуждания $1/N$. Коэффициенты взяты из таблиц \ref{tab:n_i_log_log} и \ref{tab:n_i_u_log_log}.}
\label{fig:ni_fn_vs_gNun}
\end{figure}

Логарифмический масштаб графиков представлен на рисунке \ref{fig:ni_fn_vs_gNun_log}. 
Здесь ситуация выглядит совершенно иначе: на всех графиках наблюдается расхождение $f_i$ и $g_i$ по мере сближения с нулём.
Причём теперь $g_i$ и её аппроксимация сливаются в одну кривую (что говорит о правильности полученной линеризацией функции \eqref{eq:g_n_expect}). 
Из прошлого раздела мы узнали, что асимпотические пределы $n_1$ и $n_3$ равны в пределах погрешности между выбранными зависимостями.

Однако пределы двух других функций как численно, так и графически расходятся между $f_i$ и $g_i$, что противоречит предположениям о связи зависимостей в пределе бесконечного числа шагов блуждания.


\begin{figure}
\centering
\includegraphics[width=\textwidth]{n_i_fN_vs_gNun_log.png}
\caption{Доли узлов $n_1$-$n_4$ в лог-лог масштабе как функции от количества шагов (синий пунктир) \eqref{eq:n_i_log_log}, а так же сложные функции от количества уникальных узлов от количества шагов (оранжевая линия -- прямая подстановка функций  \eqref{eq:gi_appprox2} в \eqref{eq:gi_approx1}, зелёная - аппроксимация \eqref{eq:g_n_expect}), по горизонтали -- обратное количество шагов блуждания $1/N$. Коэффициенты взяты из таблиц \ref{tab:n_i_log_log} и \ref{tab:n_i_u_log_log}.}
\label{fig:ni_fn_vs_gNun_log}
\end{figure}
\include{Sections/DS_Bulk_NunVSN}
\include{Sections/DS_Atm}
\include{Sections/DS_Atm_NunVSN}
\section{Сравнение поведения атмосфер блужданий $\pkn$ и долей узлов $n_i$ модели RW}

Формула \eqref{eq:pkn}, описывающая атмосферу $a_t$ блуждания модели RW через число соседей конца блуждания $n_end$, ранее исследованное в разделе \ref{sec:neigh}, указывает на возможное сходство свойств модели RW.
Очевидно, атмосфера характеризует локальное координационное число блуждания в его конечном узле, однако сходство необходимо искать между долей узлов с $v$ соседями и вероятностью $4-v$ атмосферы блуждания, где $v=\{1,2,3,4\}$.
В данном разделе будет проведено сравнение ранее полученных результатов для долей узлов $n_i$ (в разделе \ref{sec:neigh}) и вероятностей атмосферы $\pkn$ (в разделе \ref{sec:atm}). 
Сравнение будет проведено для каждой группы функций по отдельности: для функций от $N$ (формула \eqref{eq:n_i_log_log} и \eqref{eq:pi_N}) и от $\Nun$ (формула \eqref{eq:n_i_u_log_log} и \eqref{eq:pi_Nu})

\subsection{Сравнение функций от $\Nun$}

Рассмотрим оценки коэффициентов шкалирующих функций от $\Nun$ (таблицы \ref{tab:n_i_u_log_log} для $n_i$ и \ref{tab:p_i_u_log_log} для $\pkn$).

\begin{table}[h]
\centering
\begin{tabular}{|c|c|c|c|}
\hline
v & $d(n_v)$ & $d(p^{(4-v)})$ \\ \hline
1 & 0.015(1) & 0.097(3) \\ \hline
2 & 0.053(2) & 0.141(3) \\ \hline
3 & 0.203(2) & 0.214(6) \\ \hline
4 & 0.741(5) & 0.59(2) \\ \hline
\end{tabular}
\caption{Пределы шкалирующих функций от $\Nun$: долей узлов с $v$ соседями (центральный столбец) и вероятностей блужданий с фиксированной атмосферой $4-v$ (правый столбец) от $\Nun$ (столбцы $d$ в таблицах \ref{tab:n_i_u_log_log} для $n_i$ и \ref{tab:p_i_u_log_log} для атмосфер).}
\label{tab:n_vs_atm_d}
\end{table}

По таблице \ref{tab:n_vs_atm_d} видно, что несмотря на соразмерность пределов, в границах погрешностей они значительно отличаются.  

\begin{table}[h]
\centering
\begin{tabular}{|c|c|c|c|}
\hline
v & $s(n_v)$ & $s(p^{(4-v)})$ \\ \hline
1 & 0.479(2) & 0.310(6) \\ \hline
2 & 0.214(1) & 0.323(7) \\ \hline
3 & 0.244(2) & 0.44(4) \\ \hline
4 & 0.225(4) & 0.25(1) \\ \hline
\end{tabular}
\caption{Степенные коэффициенты шкалирующих функций от $\Nun$: долей узлов с $v$ соседями (центральный столбец) и вероятностей блужданий с фиксированной атмосферой $4-v$ (правый столбец) от $\Nun$ (столбцы $s$ в таблицах \ref{tab:n_i_u_log_log} для $n_i$ и \ref{tab:p_i_u_log_log} для атмосфер).}
\label{tab:n_vs_atm_s}
\end{table}

Сравнение степенных коэффициентов (см. таблицу \ref{tab:n_vs_atm_s}) так же не показывает какого-либо численного сходства  между величинами - все значения значительно отличаются друг друга больше чем их погрешность.

\begin{table}[h]
\centering
\begin{tabular}{|c|c|c|c|}
\hline
v & $q(n_v)$ & $q(p^{(4-v)})$ \\ \hline
1 & 0.313(1) & 0.604(5) \\ \hline
2 & 0.567(3) & 0.585(5) \\ \hline
3 & 0.542(5) & 0.52(1) \\ \hline
4 & -1.20(1) & -1.142(9) \\ \hline
\end{tabular}
\caption{Линейные коэффициенты шкалирующих функций от $\Nun$: долей узлов с $v$ соседями (центральный столбец) и вероятностей блужданий с фиксированной атмосферой $4-v$ (правый столбец) от $\Nun$ (столбцы $s$ в таблицах \ref{tab:n_i_u_log_log} для $n_i$ и \ref{tab:p_i_u_log_log} для атмосфер).}
\label{tab:n_vs_atm_q}
\end{table}

Среди линейный коэффициентов (см. таблицу \ref{tab:n_vs_atm_q}) заметно лишь поведенческое сходство в виде одинаковых знаков рассматриваемых пар коэффицентов. В остальном так же численное сходство отсутствует.

\subsection{Сравнение функций от $N$}

Далее рассмотрим оценки коэффициентов шкалирующих функций от $N$ (данные взяты из таблиц \ref{tab:n_i_log_log} для $n_i$ и \ref{tab:p_i_log_log} для $\pkn$). Аналогично прошлому разделу, сравниваться будут пары среди свободных коэффициентов или пределы функций (таблица \ref{tab:n_vs_atm_b}), степенные коэффициенты (таблица \ref{tab:n_vs_atm_a}) и, наконец, линейные коэффициенты (таблица \ref{tab:n_vs_atm_k})

\begin{table}[h]
\centering
\begin{tabular}{|c|c|c|c|}
\hline
v & $b(n_v)$ & $b(p^{(4-v)})$ \\ \hline
1 & 0.014(1) & 0.092(4) \\ \hline
2 & 0.037(2) & 0.137(4) \\ \hline
3 & 0.202(3) & 0.213(6) \\ \hline
4 & 0.759(5) & 0.62(1) \\ \hline
\end{tabular}
\caption{Пределы шкалирующих функций от $N$: долей узлов с $v$ соседями (центральный столбец) и вероятностей блужданий с фиксированной атмосферой $4-v$ (правый столбец) от $\Nun$ (столбцы $b$ в таблицах \ref{tab:n_i_log_log} для $n_i$ и \ref{tab:p_i_log_log} для атмосфер).}
\label{tab:n_vs_atm_b}
\end{table}

\begin{table}[h]
\centering
\begin{tabular}{|c|c|c|c|}
\hline
v & $a(n_v)$ & $a(p^{(4-v)})$ \\ \hline
1 & 0.417(2) & 0.259(6) \\ \hline
2 & 0.171(1) & 0.272(6) \\ \hline
3 & 0.219(3) & 0.37(3) \\ \hline
4 & 0.189(3) & 0.202(7) \\ \hline
\end{tabular}
\caption{Степенные коэффициенты шкалирующих функций от $N$: долей узлов с $v$ соседями (центральный столбец) и вероятностей блужданий с фиксированной атмосферой $4-v$ (правый столбец) от $\Nun$ (столбцы $s$ в таблицах \ref{tab:n_i_log_log} для $n_i$ и \ref{tab:p_i_log_log} для атмосфер).}
\label{tab:n_vs_atm_a}
\end{table}

\begin{table}[h]
\centering
\begin{tabular}{|c|c|c|c|}
\hline
v & $k(n_v)$ & $k(p^{(4-v)})$ \\ \hline
1 & 0.3425(8) & 0.613(5) \\ \hline
2 & 0.573(4) & 0.596(4) \\ \hline
3 & 0.588(3) & 0.54(1) \\ \hline
4 & -1.239(9) & -1.17(1) \\ \hline
\end{tabular}
\caption{Линейные коэффициенты шкалирующих функций от $N$: долей узлов с $v$ соседями (центральный столбец) и вероятностей блужданий с фиксированной атмосферой $4-v$ (правый столбец) от $N$ (столбцы $k$ в таблицах \ref{tab:n_i_log_log} для $n_i$ и \ref{tab:p_i_log_log} для атмосфер).}
\label{tab:n_vs_atm_k}
\end{table}

Сравнение функций от $N$ так же показало отсутствие численного сходства коэффициентов или какой-либо закономерности среди групп значений. 
Исключением оказалось аналогичное группе функций от $\Nun$ сходство знаков линейных коэффициентов.
По остальным возможным признакам сходства корреляция не наблюдается.

\subsection{Заключение: Итоги сравнения}

О связи между долей узлов с фиксированным числом соседей $\la n_v \ra$ и вероятностью атмосферы $p^{(4-v)_N}$ ублуждания RW существует говорит совсем немного аргументов.
Во-первых, это подтверждённый степенной характер аппроксимации обеих величин как относительно числа шагов блуждания $N$, так и относительно числа уникальных узлов $\Nun$.
Во-вторых, это схожесть знаков линейных коэффициентов. 
Однако, если она и существует, против чего говорит полное отсутствие численного сходства между коэффициентами, то крайне слабая, ввиду разной статистической мощности наблюдаемых величин. 
Очевидно, что $\la n_v \ra$ охватывает геометрическое поведение всего блуждания, в то же как $\la p^{(v)} \ra$ описывает поведение лишь на его концах, характер которых с увеличением длины блуждания становится некоррелируемым с поведением внутренних узлов.

Не подтвердилась так же и универсальность свойств локального координационного числа по отношению с SAW-модели, где пределы долей узлов и вероятностей атмосфер бесконечно-больших блужданий имели значительно большее сходство, нежели в данной модели.

\section{Критическое поведение модели IsingISAW на треугольной решётке}

В данном разделе проводится исследование критической области модели Изинга на случайном блуждания без самопересечений на треугольной решётке (далее, TrIsingISAW).
В отличие от классической модели на квадратной решётке, узлы треугольной решётки имеют две дополнительные связи по одной из диагоналей (см. рисунок \ref{fig:lattices}), 
вследствие чего координационное число данной модификации (кол-во возможных связей у одного узла) увеличено по сравнению с квадратной с 4 до 6.

\subsection{Поиск точки фазового перехода}

\begin{equation}
\label{eq:IsISAW_H}
 H_{N,u,\{\sigma\}} = -\sum_{i,j} J\sigma_i \sigma_j, i,j \in u, |u| = N
\end{equation}

\begin{equation}
\label{eq:TrIsISAW_E}
\la \epsilon \ra = \la H \ra / N
\end{equation}

Были проведены симуляции Монте-Карло при нулевом внешнем поле и $J \in [0,0.9]$. 
Итоговое количество шагов симуляций от $10^10$ до $10^11$, симулированные блуждания имеют длину $N$ от 100 до 7200.
Были собраны данные для удельной энергии системы на спин $\la \epsilon \ra$ \eqref{eq:TrIsISAW_E} и средняя 2-я и 4-я степени намагниченности на спин $\la m^2 \ra$, $\la m^4 \ra$.
Так же собрана статистика среднего расстояния между концами блуждания $R^2_N$.

\begin{figure}[h]
\begin{subfigure}{0.49\textwidth}
\includegraphics[width=\textwidth]{TrIsISAW_E.png}
\end{subfigure}
\hfill
\begin{subfigure}{0.49\textwidth}
\includegraphics[width=\textwidth]{TrIsISAW_m2.png}
\end{subfigure}
\caption{Слева: удельная энергия узла \eqref{eq:TrIsISAW_E} модели TrIsingISAW (без учёта константы $J$). Справа: средняя вторая степень намагниченности узла модели TrIsingISAW. Длины конформаций в обоих графиках от 100 до 7200 (длины отмечены разными цветами)}
\label{fig:TrIsISAW_E}
\end{figure}

На графике \ref{fig:TrIsISAW_E} показана зависимость удельной энергии \eqref{eq:TrIsISAW_E} узла модели TrIsingISAW.
График показывает что удельная энергия системы стремится к -3J в пределе бесконечной длины конформации, 
что логично, поскольку с ростом $J$ узлы приобретают наиболее возможное число связей (то есть, 6), но так как св
язи существуют между парами узлов,
необходимо поделить число связей на 2.
При малых $J$ энергия почти не зависит от длины цепочки $N$, но начиная с $J > 0.53$, расхождение графиков становится наиболее четким.

\begin{equation}
\label{eq:IsISAW_m2}
	m^{k} = (\sum_{i \in u} \sigma_i / N)^k
\end{equation}

На графике \ref{fig:TrIsISAW_m2} изображен момент намагниченности второго порядка \eqref{eq:IsISAW_m2} в зависимости от $J$.
Графики величины для конформаций разных длин имеют четкое пересечение в $J \approx 0.545$.

\begin{equation}
\label{eq:IsISAW_U4}
	U_4 = 1 - \frac{\la m^4 \ra}{3 \la m^2 \ra}
\end{equation}

\begin{figure}[h]
\begin{subfigure}{0.49\textwidth}
\includegraphics[width=\textwidth]{TrIsISAW_U4.png}
\end{subfigure}
\hfill
\begin{subfigure}{0.49\textwidth}
\includegraphics[width=\textwidth]{TrIsISAW_m2_distr.png}
\end{subfigure}
\caption{Слева: кумулянт Биндера модели TrIsingISAW с длиной конформации от 100 до 7200 (длины отмечены разными цветами).
Справа: распределение удельной намагниченности модели TrIsingISAW на конформации длиной $N=7200$ при $J \in [0.53,0.55]$ (значения J отмечены разными цветами)}
\label{fig:TrIsISAW_E}
\end{figure}


График \ref{fig:TrIsISAW_U4} показывает зависость кумулянта Биндера \eqref{eq:IsISAW_U4} от J. 
Пересечение графиков от конформаций разных длин, соответствующее переходу от парамагнетических к ферромагнетическим свойствам, снова наблюдается в $J \approx 0.545$.

Распределение значений удельной намагниченности для модели TrIsingISAW изображено на графике \ref{fig:TrIsISAW_m2_distr}.
График показывает, что распределение с преобладающими малыми по модулю значениями удельной намагниченности уступают 
распределениям с преобладающими крайними значениями намагниченности возле $J=0.545$, где распределение близко к почти равномерному.

На основании рассмотренных величин и протекаемых по ним переходов, оценка точки фазового перехода следующая:

\begin{equation}
	J_c = 0.545(5)
\end{equation}

Для оценки критического кумулянта, ввиду слишком большой наблюдаемой погрешности, требуются более длительные симуляции длин $N > 5000$.

\newpage

\subsection{Необсуждённые графики}

\begin{figure}[h]
\centering
\includegraphics[width=0.7\textwidth]{TrIsISAW_R2log.png}
\caption{Расстояние между концами блужданий конформаций от длины $N$ при $J \in [0.51,0.59]$ в логарифмическом масштабе. 
Для наглядности добавлены линии $N^{2v}$, где $v = 4/7$ (красная линия) и $v=3/4$ (чёрная линия).}
\label{fig:TrIsISAW_R2log}
\end{figure}


\section{Критическое поведение взаимодействующего блуждания без самопересечений на треугольной решётке}

Основная цель данного раздела - исследование критических свойств модели ISAW на треугольной решётке (далее, TrISAW).
Данная задача решалась ранее, в статье \cite{Privman1986}, использованием приближений наблюдаемых величин рядями Тейлора.
Полученная оценка точки фазового перехода для TrISAW выписана на таблице \ref{tab:crits}.

В данном разделе мы воспользуемся известными данными о шкалировании радиуса между краями блуждания $R^2_N$:
на основании результатов о невзаимодействующем блуждании без самопересечений \cite{Rensburg2015}, а так же о критическом поведении взаимодействующего блуждания \cite{Duplantier1987} на квадратной решётке,
найдём область конформационного перехода модели и, тем самым, уточним оценку точки фазового перехода.
 

\begin{figure}[h]
\begin{subfigure}{0.49\textwidth}
\includegraphics[width=\textwidth]{TrISAW_R2log.png}
\caption{}
\label{fig:TrISAW_R2log}
\end{subfigure}
\hfill
\begin{subfigure}{0.49\textwidth}
\includegraphics[width=\textwidth]{TrISAW_R2toN2v.png}
\caption{}
\label{fig:TrISAW_R2toN2v}
\end{subfigure}
\caption{Слева: Расстояние между концами блужданий длины $N$ при $J \in [0.51,0.59]$ в логарифмическом масштабе. 
Для наглядности добавлены линии $N^{2\nu}$, где $\nu = 4/7$ (красная линия) и $\nu=3/4$ (чёрная линия).
Справа: отношение расстояния между концами блуждания $R^2$ к $N^{2\nu}$, где $\nu=4/7$ при $J \in [0.52,0.58]$}
\end{figure}

\section{Заключение}

В ходе первой части работы было исследовано поведение локального координационного числа модели Изинга на случайном блуждании без самоперечений, а так же в модели взаимодействующего блуждания.
Плотность связей была рассмотрена в модели на нескольких решётках в зависимости от константы J,
а так же в состоянии нулевого взаимодействия в зависимости от длины цепочки.
Методами Монте-Карло были определены характер зависимости средних долей узлов с фиксированным числом соседей от длины конформации и пределы долей на бесконечно большой длине цепочки, а так же проведено сравнение этих значений на предмет универсальности среди квадратной, треугольной, кубической и 4D-гиперкубической решёток.

Для проверки универсальности свойства средних долей узлов с фиксированным числом соседей между внутренними узлами и концами конформаций, было проведено сравнение полученных для Ising-ISAW на квадратной решётке пределов средних долей при бесконечно большой длине в точке J=0 с предельными значениями вероятностей атмосфер невзаимодействующего блуждания из работы \cite{owczarek2008scaling}. Результат показал явную разницу поведения внутреннего и граничного узла конформаций с ростом длины конформации.

Аналогичное исследование локального координацинного числа было проведено для простого случайного блуждания на квадратной решётке.
Результаты симуляций показали степенной характер шкалирования долей узлов с фиксированным числом соседей как от числа шагов случайного блуждания, 
так и от числа уникальных узлов блуждания.
Похожим свойством обладало и атмосфера простого случайного блуждания, несмотря на гораздо большее численное отличие шкалирующих показателей между величинами.

Другим важным направлением являлось исследование критических свойств моделей Ising-ISAW и ISAW на треугольных решётках:
были найдены и уточнены оценки точек фазового перехода моделей, а так же проверена применимость ранее найденных критических показателей для квадратных решёток в треугольных модификациях моделей. 
\section{Программно-техническое приложение}

В данном разделе будут описаны особенности работы с суперкомпьютером НИУ ВШЭ, которые могут быть важными дополнением к основной инструкции пользователя.

\subsection{Применение jit-компиляции при программировании на языке Python}
\label{subsection:njit_problem}

Симуляции случайного блуждания с самопересечениями (для кода см. папку $Random\_Walk$  \cite{web:ProjectMagnetRepos}) были запрограммированны на языке Python с компиляцией с помощью пакета numba метод jit. В качестве окружения была использована стандартная библиотека $Python/Anaconda\_v11.2021$ встроенная в стандартное ПО суперкомпьютера. 

Выполнение первых экспериментов по симуляциям шло крайне медленно - результаты за семь дней можно увидеть на таблице \ref{tab:Ran_Walk_neigh_1}

\begin{table}[h]
    \centering
    \begin{tabular}{|c|c|c|c|c|c|c|}
        \hline
        N & $steps$ & $unique$ & $n_{1}$ & $n_{2}$ & $n_{3}$ & $n_{4}$ \\ \hline
        100 & 7450000 & 0.49(8) & 0.07(3) & 0.33(9) & 0.36(7) & 0.24(9) \\ \hline
        200 & 5684000 & 0.44(7) & 0.05(2) & 0.29(7) & 0.35(5) & 0.30(9) \\ \hline
        500 & 2045000 & 0.39(6) & 0.04(1) & 0.24(5) & 0.34(4) & 0.38(8) \\ \hline
        1000 & 654000 & 0.36(5) & 0.03(1) & 0.22(4) & 0.33(4) & 0.42(7) \\ \hline
        2500 & 132000 & 0.33(4) & 0.027(7) & 0.19(3) & 0.31(3) & 0.48(6)  \\ \hline
        5000 & 37000 & 0.31(4) & 0.024(5) & 0.17(3) & 0.29(3) & 0.51(6) \\ \hline
        10000 & 10000 & 0.29(3) & 0.021(4) & 0.16(2) & 0.28(3) & 0.54(5) \\ \hline
    \end{tabular}
    \caption{Средние доли узлов c 1-4-мя соседями в конформациях модели Random-Walk длин $10^{2}-10^{4}$}
    \label{tab:Ran_Walk_neigh_1}
\end{table}

Для сравнения с другими платформами, в случае длины цепочки $N=10000$, процесс из 10000 шагов на Google Colab занимал не более 7 часов.

\begin{figure}[h]
    \centering
    \includegraphics[width=0.95\textwidth]{experiment_time.png}
\end{figure}

Решением проблемы оказалось создание собственного окружения с другими версиями используемых пакетов numpy и numba (полный список так же есть в репозитории с кодом \cite{web:ProjectMagnetRepos}). Новые результаты за 7 дней описаны в продолжении основного раздела.

При обсуждении столь значительного различия во времени выполнениями между окружениями поддержкой было выдвинуто предположение, что окружения отличаются сторонними библиотеками линейной алгебры, используемой пакетом numpy: наиболее распространенными считаются OpenBLAS и Intel MKL. Основным фактором преимущества той или иной библиотеки является именно процессор (Intel или non-Intel). 

В новом окружении пакетом numpy использовалась именно библиотека OpenBLAS, в то время как в Anaconda - Intel MKL. Это следовало из применения в данных окружениях следующего:

\begin{code}
import numpy

print(numpy.show\_config())
\end{code}

Подробнее об определении какая библиотека линейной алгебры используется в пакете numpy можно найти \href{https://shaalltime.medium.com/benchmark-numpy-with-openblas-and-mkl-library-on-amd-ryzen-3950x-cpu-96184f91057f}{здесь}. 


\subsection{Итерации программного комплекса Rand-Walk}

Подраздел посвящён описанию версий программного комплекса для симуляций модели простого случайного блуждания фиксированной длины N на квадратной решётке.
(для кода см. папку $Random\_Walk$  \cite{web:ProjectMagnetRepos})

\begin{enumerate}
\item \textbf{Drunken\_Sailor\_def.py} - базовый алгоритм симуляций, предназначенный для проверки работы основных функций: 
\begin{itemize}
\item experiment - генерация цепочки и подсчёт наблюдаемых (доли узлов с числом соседей 1-4, а так же доля уникальных узлов цепочки)
\item \textit{complex\_experiment} - запись результирующего массива для одной цепочки (experiment) и набора цепочек (шаг - кол-во опытов между анализом данных)
\item \textit{write\_results} - запись текущих результатов (средних наблюдаемых по всем экспериментам) в текстовый файл
\item \textit{save\_distr} - распределение значений наблюдаемых по всем экспериментам
\item \textit{save\_history} - сохранение истории средний значений для анализа сходимости результаотв симуляций
\end{itemize}
Цепочка генерируется как двумерный массив точек, потому наиболее его медленной частью является поиск уникальных узлов цепочки через \textit{np.unique}, не поддерживающий njit-комплиляцию при обработке двумерного массива.
\item \textbf{Drunken\_Sailor.py} - первая версия симуляционного комплекса с jit-компилируемой частью. Алгоритмически не отличается от \textbf{Drunken\_Sailor\_def.py}, но значительно быстрее базовой версии
\item \textbf{Drunken\_Sailor\_v2.py} - оптимизированная версия \textbf{Drunken\_Sailor.py} c расширенной njit-комплиляцией:
\begin{itemize} 
\item \textit{create\_walk} - генерация цепочки как массива поворотов блуждания начиная с начальной точки $(0,0)$, затем - как массив всех точек блуждания
\item \textit{calc\_fractions} - основная функция подсчёта наблюдаемых. Так же модифицирована над подсчёт атмосферы каждого блуждания
\item В \textit{complex\_experiment} добавлено распараллеливание проведение набора экспериментов за шаг между выводом данных, что позволило значительно ускорить работу комплекса.
\item \textit{stats} - подсчёт текущего результата для наблюдаемых долей
\item \textit{atm\_bins} - подсчёт долей блужданий с атмосферой 0-3
\end{itemize}
\end{enumerate}


\bibliographystyle{plain}
\bibliography{bibliography}

%\fi

\end{document}



