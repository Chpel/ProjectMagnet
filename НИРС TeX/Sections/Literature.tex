\section{Литературный обзор}

С целью поиска информации о локальном координационном числе (что в случае блужданий может также быть названо числом соседей узла), был проведён обзор литературы, возможно имеющей отношение к рассматриваемым в рамках проекта моделей.

\subsection{Livne, Meirovich: Polymers Adsorbed on a surface}

\subsubsection{Особенности модели блуждания}

В работе \cite{LivneSAW1988} исследуется поведение адсорбирующего случайного блуждания без самопересечений на кубической решётке со следующими особенностями симуляции

\begin{itemize}
    \item Случайное блуждание длины N+1  строится пошагово (N+1 мономеров в цепочке или N шагов), из начала координат (x=0, y=0, z=0) с ограничением на верхнее полупространство (то есть, z >= 0 и плоскость z=0 имеет открытые граничные условия).
    \item Энергия конформации считается как число мономеров, лежащих на поверхности (у которых $z_{i} = 0$), умноженное на константу взаимодействия полимера и поверхности $\epsilon$
    \item Вероятность i-й конформации считается последовательно: вводится новая статсумма, суммирующая для заданного направления текущей недостроенной цепочки всевозможные хвосты остаточной длины (10)\cite{LivneSAW1988}. 
\end{itemize}

\subsubsection{Подробнее о статсумме и методе Сканирования }

В данном подразделе вольным образом объясняется действие статсуммы, созданное методом сканированния. Так как при симуляции строится новое блуждание "c нуля", требуется оценка вероятности как каждого шага (точнее, направления $v_{k}$) так и всего блуждания.

Поэтому для k-го шага вероятность рассчитывается следующим образом:

\begin{enumerate}
    \item Считается статсумма куска будущего блуждания из b ($<= N - k + 1$) шагов, начинающая с направления v на высоте $z_{k-1}$:
    
    \begin{equation}
        Z_{k}(v, b, z_{k-1}, v_{k-1}) = \sum_{j}\exp{(-\epsilon m_{j}(0)/k_{b}T)}
        \label{Z_Lenvi}
    \end{equation}
     
    \item Затем проводится расчёт вероятности выбрать направление v из всех возможных на k-м шаге:
    
    \begin{equation}
        p_{k}(v|b,z_{k-1},v_{k-1}) = Z_{k}(v, b, z_{k-1}, v_{k-1}) / \sum_{v} Z_{k}(v, b, z_{k-1}, v_{k-1})
        \label{p_k_Lenvi}
    \end{equation}
    
    \item Итоговой вероятностью всего построения будет произведение всех вероятностей каждого шага по выбранным направлениям:
    
    \begin{equation}
        P_i(b) = \prod_{k=1}^{N} p_{k}(v_{k}|b,z_{k-1},v_{k-1})
    \end{equation}
\end{enumerate}

\subsubsection{Результаты работы}

Основными итогами работы являлось подтверждение эффективности метода "сканирования" для работы с длинными цепочками в модели адсорбирующего блуждания, определено критическое шкалирование перпердикулярного радиуса инерции (радиуса инерции проекции блуждания на ось z), а также профиля мономерной концентрации $p(z)$ (средняя доля узлов конформации длины N+1 на фиксированной высоте z от поверхности).

Информации о локальном координационном числе в статье найдено не было.

\newpage

\subsection{Madras, Sokal: The Pivot Algorithm}

Работа \cite{madras1988pivot} повествует о работе и эффективности алгоритма Пивота в изучении модели случайного блуждания без самопересечений (СБС).

\subsubsection{Основные принципы алгоритма}

Каждый шаг алгоритма проводит следующие действия над уже сгенерированной цепочкой длины N+1:

\begin{itemize}
    \item Случайно выбирается с равномерным распределением для рассматриваемых узлов $p_{k} = 1/N$ k-й узел цепочки ($0 <= k <= N-1$, хотя начальную точку k=0 на практике не используют)
    \item Последующую половину цепочки ($\omega_{k+1}, \omega_{k+2},\dots,\omega_{N}$ заменяют элементов группы симметрии (проще говоря, отражают, поворачивают или проводят комбинацию этих действий)
    \item В случае, если полученная операцией цепочка осталась без самопересечений, шаг принимается - в противном случае, шаг производится заново
\end{itemize}

В статье так же была доказана эргодичность алгоритма, а так же средние вероятности принятия каждого из возможных преобразований.

Для симуляций в качестве стартовой позиции использовалось два варианта: прямые цепочки ''rods'', при которых проволилось некоторое кол-во шагов до достижения термального равновесия системы (в таком состоянии процесс из следующих состояний цепочки становится близким по расспределению к стационарному стохастическому), или же ''димеризованные цепочки'' , состояние которых уже считается равновесным. Второй метод становится крайне времезатратным при большой длине цепочки, поэтому при N>=2400 чаще применялась термолизация прямых цепочек.

Пристальное внимание в статье было обращено к среднему радиусу инерции $S^{2}_{N}$ и квадрату расстояния между концами $\omega^{2}_{N}$, а так же к оценке метрической экспоненты $\upsilon$, характеризующей обе величины в крит. области модели: 

\begin{align*}
    \la \omega^{2}_{N} \ra &\sim N^{2\upsilon} \\
    \la S^{2}_{N} \ra &\sim N^{2\upsilon} 
\end{align*}

В оценке будущей работы было так же отмечено, что алгоритм Пивота не подходит для расчёта связующей $\mu$ и критической $\gamma$ экспонент (связующую константу так же называют \textit{эффективным координационным числом}), так как алгоритм алгоритм работает лишь в случае канонического ансамбля (при фиксированной длине цепочки) и требуется алгоритм, работающий уже в большом каноническом ансамбле (с цепочками изменяемой длины).

В статье не рассматривалось как таковое ''число соседей узлов''.


\subsection{Спицер: Теория случайных блужданий}

Основным объектом исследования в книге \cite{Spitser1969} является класс случайных блужданий, допускающих пересечение. 
В их число входит интересовавшее в рамках летней производственной практики простое случайное блуждание на двумерной решётке.


Первые две главы являлись ознакомительными, как с самим классом случайных блужданий и их свойств, так и с аппаратом исследования их поведенния способом гармонического анализа.
В наиболее привлекшей внимание главе №3 косвенным образом рассматривалось свойство локального координационного числа одного из концов блуждания бесконечной длины - оно было представлено в задаче 3.9.


Полный обзор книги и в особенности главы №3 можно прочесть в отчёте о проведении летней производственной практики.
