\section{Введение}

Модель случайных блужданий без самопересечений (далее - СБС) - одна из наиболее широко изученных моделей из класса линейных полимеров. 
Более того, она является одной из простейших моделей для изучения критического поведения - так, в случае когда модель усложнена наличием взаимодействия между ближайшими узлами цепочки, её фазовый переход оказывается зафиксирован между состояниями её растворителя, и при термическом равновесии системы полученная полимерная цепочка будет схлопнутой в условиях сильного растворителя или вытянутой в слабом растворителе.
Трикритичность данного перехода была описана в работе \cite{Gennes1979}.

Влияние близких связей было широко рассмотрено среди класса моделей магнитного полимера, взаимодействие между узлами которых стало ещё более сложным:
каждый узел обладает спином, а сила взаимодействия между ближайшими узлами стала отдельным параметром. 
Её название - модель Изинга на случайном блуждании без самопересечений.
В работе \cite{Garel1999} был рассмотрен случай, когда она так же обладает переменным внешним полем, и все заключения о её магнитных свойствах были оценены в сравнении с моделью среднего поля.
В то же время влияние геометрических свойств модели на магнитные не до конца ясно и их изучение в некоторых случаях требует статистического подхода.

В предыдущей работе \cite{faizullina2021critical} было определено, что фазовый переход двумерной модели Изинга на CБС
имеет непрерывный характер. 
В этой работе мы продолжаем изучать геометрические свойства данной модели и сравнивать их с её родительскими моделями или модификациями, такими как классическая модель Изинга на регулярной решётке, опредёленная в работе \cite{selke2006critical}, и взаимодействующее случайное блуждание без самопересечений в соответствующих критических областях. 
Мы предполагаем, что модели с похожими геометрическими свойствами будут так же иметь схожесть в магнитных, что мы рассмотрим при сравнении кумулянтов биндера в области $\theta$-перехода моделей при равных значениях асферичности.
Также могут быть интересны для рассмотрения решётки, на которых исследуются конформации модели, как параметр задающий закон, по которому определяется близость узлов и следовательно - существование тех или иных связей между ними в цепочке. 
Данное направление было начато в частном случае среди концов случайных блужданий без самопересечений на квадратной решётке в работе \cite{owczarek2008scaling}. 
В нашей работе мы рассмотрим эти результаты с долей узлов с фиксированным количество соседей на квадратной решётке, как обобщение на внутренние узлы цепочки, а так же рассмотрим поведение данного геометрического свойства среди разных решёток в пределе бесконечной длины цепочки.  

\section{Модели и методы}
В рамках данной работы определяется несколько моделей: первой будет модель Изинга на случайном блужданий без самопересечений (далее - Ising-ISAW). Энергия системы конформации $u$ (последовательности точек на решётке, на которых размещёна цепочка) фиксированной длины N с последовательностью спинов в узлах s, принимающих значение $+1$ или $-1$, рассчитывается как сумма взаимодействий между ближайшими узлами цепочки:

\begin{equation}
E(s,u) = -J \sum_{\la i, j \ra} s_i s_j,\ \ \ \ i,j \in u, |u|=N
\end{equation}
Статическая сумма модели берётся по всем возможным последовательностяv ${s}$ и конформациям $u$:

\begin{equation}
Z = \sum_s \sum_u \exp{(\frac{-E}{kT})}
\end{equation}

где $T$ — температура, $k$ — постоянная Больцмана. Без потери общности можно считать $kT = 1$, тем самым оставляя J самостоятелньым параметром модели.
В первой части работы модель Ising-ISAW рассматривается только на квадратной решётке, где соседями узла можно считать мономеры, расположенные сверху, снизу, слева и справа от него.

Под "родительскими" к Ising-ISAW моделями определим следующие две: с одной стороны, взаимодействующая составляющая модели берётся из классических моделей Изинга - в частности, мы будем рассматривать классическую модель Изинга на регулярной прямоугольной решётке (далее - прямоугольный Изинг), определеную так же в работе \cite{selke2006critical}.
В ней узлы со спинами заполняют всю решётку со стороной $L=N_x$ и отношением сторон $r=\frac{N_y}{N_x}$. Длина стороны считается как количество узлов решётки в одном ряде. 
Решётка может иметь как периодический граничные условия - когда узлы на противоположных краях решётки считаются соседними - так 
Тогда энергия системы с последовательностью спинов $s$, рассчитывается как сумма взаимодействий между ближайшими узлами по всей решётке:

\begin{equation}
E(L, r, \{s\}) = -J \sum_{\la i, j \ra} s_i s_j,\ \ \ \ i,j = (x_i, u_i), (x_j, y_j) \in [1..L] \times [1..L*r]
\end{equation}

Статическая сумма модели берётся только возможным последовательностяv ${s}$:

\begin{equation}
Z = \sum_s \exp{(\frac{-E}{kT})}
\end{equation}

В качестве сравниваемого между моделями магнитного свойства определим кумулянт Биндера (или критический кумулянт):

\begin{equation}
\label{eq:Cumulant}
U_{4} = 1 - \frac{\la m^{4} \ra}{3 (m^{2})^{2}}
\end{equation}

где $\la m^{2} \ra$  - средний квадрат удельной намагниченности, $\la m^{4} \ra$ - средная удельная намагниченность в четвертой степени. 

С другой стороны, определелим модель взаимодействующего блуждания без самопересечений (далее - ISAW) на квадратной решётке из работы \cite{caracciolo2011geometrical}.
В отличие от Ising-ISAW, узлы конформации не имеют спинов, и энергия модели рассчитывается как сумма связей переменной силы J между узлами:

\begin{equation}
E(\{u\}) = \sum_{\la i, j \ra} 1,\ \ \ i,j \in u, |u|=N
\end{equation}

\begin{equation}
Z = \sum_s \exp{(- \beta E(\{s\}))}
\end{equation}

В рамках работы \cite{caracciolo2011geometrical} исследовалось поведение геометрических свойств в критической области - в частности, асферичности конформации, показателя отличия системы узлов от круга.
Для этого определим показатели формы системы, такие как тензор вращения системы - матрица корреляции координат системы из N точек $w_i = (w_{i,\alpha}, w_{i,\beta})$:

\begin{equation}\label{eq:Ten_G1}
    Q_{N,\alpha\beta} = \frac{1}{N+1} \sum^{N}_{i=0}(w_{i,\alpha} - w_{c, \alpha})(w_{i,\beta} - w_{c, \beta})
\end{equation}

где $w_{c,\alpha} - \alpha$ -я координата вектора центра масс. В случае, если начало координат расположено в центре масс (следовательно, сумма векторов точек блуждания = 0), формула $\alpha\beta-$элемента тензора упрощается и численно равна второму моменту координаты (если $\alpha = \beta$), или до среднего произведения разных координат по всем точкам блуждания.

\begin{align}\label{eq:Ten_G_C}
    Q_{N,\alpha\beta} = &\frac{1}{(N+1)} \sum_{i=0}^{N} w_{i, \alpha} w_{i, \beta} \\
    \sum^{N}_{i=0}w_{i} &= 0
\end{align}

Собственные значения $q_1, q_2$ полученного тензора вращения можно интерпретировать как квадраты длин полуосей эллипса вращения системы.
Их отношение для системы длины $N$ равна:

\begin{equation}
r = \sqrt{\frac{\la q_1 \ra_N}{\la q_2 \ra_N}}
\end{equation}

Так же эти значения используются для расчёта ещё одного показателя формы -
средней асферичности:

\begin{equation}
\label{eq:Asphericity}
    \mathcal{A} = \left\langle \frac{(q_{1} - q_{2})^{2}}{(q_{1} + q_{2})^{2}} \right\rangle_{N}
\end{equation}

Модификации исследуемой системы Ising-ISAW имеют изменения в выбранной решётке - то есть, законов, по кото
: в работе




