\section{Сравнение поведения атмосфер блужданий $\pkn$ и долей узлов $n_i$ модели RW}

Формула \eqref{eq:pkn}, описывающая атмосферу $a_t$ блуждания модели RW через число соседей конца блуждания $n_end$, ранее исследованное в разделе \ref{sec:neigh}, указывает на возможное сходство свойств модели RW.
Очевидно, атмосфера характеризует локальное координационное число блуждания в его конечном узле, однако сходство необходимо искать между долей узлов с $v$ соседями и вероятностью $4-v$ атмосферы блуждания, где $v=\{1,2,3,4\}$.
В данном разделе будет проведено сравнение ранее полученных результатов для долей узлов $n_i$ (в разделе \ref{sec:neigh}) и вероятностей атмосферы $\pkn$ (в разделе \ref{sec:atm}). 
Сравнение будет проведено для каждой группы функций по отдельности: для функций от $N$ (формула \eqref{eq:n_i_log_log} и \eqref{eq:pi_N}) и от $\Nun$ (формула \eqref{eq:n_i_u_log_log} и \eqref{eq:pi_Nu})

\subsection{Сравнение функций от $\Nun$}

Рассмотрим оценки коэффициентов шкалирующих функций от $\Nun$ (таблицы \ref{tab:n_i_u_log_log} для $n_i$ и \ref{tab:p_i_u_log_log} для $\pkn$).

\begin{table}[h]
\centering
\begin{tabular}{|c|c|c|c|}
\hline
v & $d(n_v)$ & $d(p^{(4-v)})$ \\ \hline
1 & 0.015(1) & 0.097(3) \\ \hline
2 & 0.053(2) & 0.141(3) \\ \hline
3 & 0.203(2) & 0.214(6) \\ \hline
4 & 0.741(5) & 0.59(2) \\ \hline
\end{tabular}
\caption{Пределы шкалирующих функций от $\Nun$: долей узлов с $v$ соседями (центральный столбец) и вероятностей блужданий с фиксированной атмосферой $4-v$ (правый столбец) от $\Nun$ (столбцы $d$ в таблицах \ref{tab:n_i_u_log_log} для $n_i$ и \ref{tab:p_i_u_log_log} для атмосфер).}
\label{tab:n_vs_atm_d}
\end{table}

По таблице \ref{tab:n_vs_atm_d} видно, что несмотря на соразмерность пределов, в границах погрешностей они значительно отличаются.  

\begin{table}[h]
\centering
\begin{tabular}{|c|c|c|c|}
\hline
v & $s(n_v)$ & $s(p^{(4-v)})$ \\ \hline
1 & 0.479(2) & 0.310(6) \\ \hline
2 & 0.214(1) & 0.323(7) \\ \hline
3 & 0.244(2) & 0.44(4) \\ \hline
4 & 0.225(4) & 0.25(1) \\ \hline
\end{tabular}
\caption{Степенные коэффициенты шкалирующих функций от $\Nun$: долей узлов с $v$ соседями (центральный столбец) и вероятностей блужданий с фиксированной атмосферой $4-v$ (правый столбец) от $\Nun$ (столбцы $s$ в таблицах \ref{tab:n_i_u_log_log} для $n_i$ и \ref{tab:p_i_u_log_log} для атмосфер).}
\label{tab:n_vs_atm_s}
\end{table}

Сравнение степенных коэффициентов (см. таблицу \ref{tab:n_vs_atm_s}) так же не показывает какого-либо численного сходства  между величинами - все значения значительно отличаются друг друга больше чем их погрешность.

\begin{table}[h]
\centering
\begin{tabular}{|c|c|c|c|}
\hline
v & $q(n_v)$ & $q(p^{(4-v)})$ \\ \hline
1 & 0.313(1) & 0.604(5) \\ \hline
2 & 0.567(3) & 0.585(5) \\ \hline
3 & 0.542(5) & 0.52(1) \\ \hline
4 & -1.20(1) & -1.142(9) \\ \hline
\end{tabular}
\caption{Линейные коэффициенты шкалирующих функций от $\Nun$: долей узлов с $v$ соседями (центральный столбец) и вероятностей блужданий с фиксированной атмосферой $4-v$ (правый столбец) от $\Nun$ (столбцы $s$ в таблицах \ref{tab:n_i_u_log_log} для $n_i$ и \ref{tab:p_i_u_log_log} для атмосфер).}
\label{tab:n_vs_atm_q}
\end{table}

Среди линейный коэффициентов (см. таблицу \ref{tab:n_vs_atm_q}) заметно лишь поведенческое сходство в виде одинаковых знаков рассматриваемых пар коэффицентов. В остальном так же численное сходство отсутствует.

\subsection{Сравнение функций от $N$}

Далее рассмотрим оценки коэффициентов шкалирующих функций от $N$ (данные взяты из таблиц \ref{tab:n_i_log_log} для $n_i$ и \ref{tab:p_i_log_log} для $\pkn$). Аналогично прошлому разделу, сравниваться будут пары среди свободных коэффициентов или пределы функций (таблица \ref{tab:n_vs_atm_b}), степенные коэффициенты (таблица \ref{tab:n_vs_atm_a}) и, наконец, линейные коэффициенты (таблица \ref{tab:n_vs_atm_k})

\begin{table}[h]
\centering
\begin{tabular}{|c|c|c|c|}
\hline
v & $b(n_v)$ & $b(p^{(4-v)})$ \\ \hline
1 & 0.014(1) & 0.092(4) \\ \hline
2 & 0.037(2) & 0.137(4) \\ \hline
3 & 0.202(3) & 0.213(6) \\ \hline
4 & 0.759(5) & 0.62(1) \\ \hline
\end{tabular}
\caption{Пределы шкалирующих функций от $N$: долей узлов с $v$ соседями (центральный столбец) и вероятностей блужданий с фиксированной атмосферой $4-v$ (правый столбец) от $\Nun$ (столбцы $b$ в таблицах \ref{tab:n_i_log_log} для $n_i$ и \ref{tab:p_i_log_log} для атмосфер).}
\label{tab:n_vs_atm_b}
\end{table}

\begin{table}[h]
\centering
\begin{tabular}{|c|c|c|c|}
\hline
v & $a(n_v)$ & $a(p^{(4-v)})$ \\ \hline
1 & 0.417(2) & 0.259(6) \\ \hline
2 & 0.171(1) & 0.272(6) \\ \hline
3 & 0.219(3) & 0.37(3) \\ \hline
4 & 0.189(3) & 0.202(7) \\ \hline
\end{tabular}
\caption{Степенные коэффициенты шкалирующих функций от $N$: долей узлов с $v$ соседями (центральный столбец) и вероятностей блужданий с фиксированной атмосферой $4-v$ (правый столбец) от $\Nun$ (столбцы $s$ в таблицах \ref{tab:n_i_log_log} для $n_i$ и \ref{tab:p_i_log_log} для атмосфер).}
\label{tab:n_vs_atm_a}
\end{table}

\begin{table}[h]
\centering
\begin{tabular}{|c|c|c|c|}
\hline
v & $k(n_v)$ & $k(p^{(4-v)})$ \\ \hline
1 & 0.3425(8) & 0.613(5) \\ \hline
2 & 0.573(4) & 0.596(4) \\ \hline
3 & 0.588(3) & 0.54(1) \\ \hline
4 & -1.239(9) & -1.17(1) \\ \hline
\end{tabular}
\caption{Линейные коэффициенты шкалирующих функций от $N$: долей узлов с $v$ соседями (центральный столбец) и вероятностей блужданий с фиксированной атмосферой $4-v$ (правый столбец) от $N$ (столбцы $k$ в таблицах \ref{tab:n_i_log_log} для $n_i$ и \ref{tab:p_i_log_log} для атмосфер).}
\label{tab:n_vs_atm_k}
\end{table}

Сравнение функций от $N$ так же показало отсутствие численного сходства коэффициентов или какой-либо закономерности среди групп значений. 
Исключением оказалось аналогичное группе функций от $\Nun$ сходство знаков линейных коэффициентов.
По остальным возможным признакам сходства корреляция не наблюдается.

\subsection{Заключение: Итоги сравнения}

О связи между долей узлов с фиксированным числом соседей $\la n_v \ra$ и вероятностью атмосферы $p^{(4-v)_N}$ ублуждания RW существует говорит совсем немного аргументов.
Во-первых, это подтверждённый степенной характер аппроксимации обеих величин как относительно числа шагов блуждания $N$, так и относительно числа уникальных узлов $\Nun$.
Во-вторых, это схожесть знаков линейных коэффициентов. 
Однако, если она и существует, против чего говорит полное отсутствие численного сходства между коэффициентами, то крайне слабая, ввиду разной статистической мощности наблюдаемых величин. 
Очевидно, что $\la n_v \ra$ охватывает геометрическое поведение всего блуждания, в то же как $\la p^{(v)} \ra$ описывает поведение лишь на его концах, характер которых с увеличением длины блуждания становится некоррелируемым с поведением внутренних узлов.

Не подтвердилась так же и универсальность свойств локального координационного числа по отношению с SAW-модели, где пределы долей узлов и вероятностей атмосфер бесконечно-больших блужданий имели значительно большее сходство, нежели в данной модели.