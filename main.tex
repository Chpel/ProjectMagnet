\documentclass{article}

\usepackage{cmap}
\usepackage[utf8]{inputenc}
\usepackage[T1]{fontenc}
\usepackage[english, russian]{babel}


%%% Дополнительная работа с математикой
\usepackage{amsmath,amsfonts,amssymb,amsthm,mathtools} % AMS
\usepackage{icomma} % "Умная" запятая: $0,2$ --- число, $0, 2$ --- перечисление

\usepackage{geometry} % Простой способ задавать поля
	\geometry{top=20mm}
	\geometry{bottom=30mm}
	\geometry{left=20mm}
	\geometry{right=20mm}

%% Номера формул
%\mathtoolsset{showonlyrefs=true} % Показывать номера только у тех формул, на которые есть \eqref{} в тексте.

%% Шрифты
\usepackage{euscript}	 % Шрифт Евклид
\usepackage{mathrsfs} % Красивый матшрифт

%% Свои команды
\DeclareMathOperator{\sgn}{\mathop{sgn}}

%% Перенос знаков в формулах (по Львовскому)
\newcommand*{\hm}[1]{#1\nobreak\discretionary{}
{\hbox{$\mathsurround=0pt #1$}}{}}

\title{Проект 20-21}
\author{Ilya Pchelintsev}
\date{}

\usepackage{graphicx}
\usepackage{cite}
\usepackage{csquotes}

\newcommand{\bj}{\beta J}
\newcommand{\dbj}{2 \bj}
\newcommand{\bh}{\beta h}
\newcommand{\lp}{\lambda_{+}}
\newcommand{\lm}{\lambda_{-}}
\newcommand{\lpm}{\lambda_{\pm}}
\newcommand{\lpd}{\lambda_{+2}}
\newcommand{\lmd}{\lambda_{-2}}
\newcommand{\lpmd}{\lambda_{\pm2}}
\newcommand{\ap}{A_{+}}
\newcommand{\am}{A_{-}}
\newcommand{\apm}{A_{\pm}}
\newcommand{\pbc}{_{PBC}}
\newcommand{\obc}{_{OBC}}
\newcommand{\dzdb}{\frac{\partial Z}{\partial \beta}}
\newcommand{\prpb}{^{'}_{\beta}}
\newcommand{\vprpb}{^{''}_{\beta}}
\newcommand{\la}{\langle}
\newcommand{\ra}{\rangle}

\numberwithin{equation}{section}

\begin{document}

\maketitle

\section{Введение}

\subsection{Одномерная модель Изинга}

Модель Изинга представляет собой решетку, в узлах которой расположены магнитные моменты, направленные "вверх"  или "вниз" , чему соответствует значение "cпина" на j-ом месте в решетке.

\[ \sigma_{j} = \pm1 \]

Энергией взаимодействия внешнего поля с моделью будем считать сумму взаимодействий поля \textit{h} с каждым из N моментов со спином $\sigma_{j}$

\[ H_{h} = - \sum_{j = 1}^{N} h  \sigma_{j} \]

Внутренним взаимодействией между двумя соседними моментами считаем:

\[ H_{J} = - \sum_{(i,j)} J  \sigma_{i}  \sigma_{j} \]

Тогда Гамильтонианом системы будет:

\begin{equation}\label{eq:Ham} 
   H = - h\sum_{j = 1}^{N}  \sigma_{j} - J \sum_{(i,j)} \sigma_{i}  \sigma_{j} 
\end{equation}

\subsection{Cтатсумма цепи Изинга общего случая ($h, J \neq 0 $) : периодич. гран. условия и Трансфер-матрица}

Для поиска решения данного случая воспользуемся методом \textbf{трансфер-матриц}.

Для начала перепишем формулу \eqref{eq:Ham} в другой форме:
\begin{equation}\label{eq:HamTrM}
    H = - \frac{h}{2}\sum_{j = 1}^{N}  (\sigma_{j} + \sigma_{j+1}) - J \sum_{j = 1}^{N} \sigma_{j}  \sigma_{j+1} 
\end{equation}

Учитывая периодические гран. условия ($\sigma_{N+1} = \sigma_{1}$), то формулы \eqref{eq:Ham} и \eqref{eq:HamTrM} тождественно равны.

Тогда статсумма такой модели будет равна:

\begin{equation}\label{ZTrM}
    Z = \sum_{\sigma} e^{-\beta H} = \sum_{\sigma} \prod_{j=1}^{N} \exp{(\bj\sigma_{j}\sigma_{j+1} + \frac{1}{2}\bh(\sigma_{j} + \sigma_{j+1}))} = \sum_{\sigma} \prod_{j=1}^{N} T(\sigma_{j}, \sigma_{j+1})
\end{equation}
    
Где $T(\sigma_{j}, \sigma_{j+1})$ - трансфер-матрица для двух соседних моментов. Поскольку один момент принимают лишь два значения ($\pm1$), а пара - уже четыре - (1,1),(1,-1),(-1,1),(-1,-1) - то, их матрица представляет с собой матрицу с элементами, соответствующими этим парам значений:
\begin{align}\label{TrM}
  T(\sigma_{j}, \sigma_{j+1}) = &
  \begin{pmatrix}
    \exp{(\bj+\bh)} & \exp(-\bj) \\
    \exp{(-\bj)} & \exp{(\bj-\bh)}
  \end{pmatrix}  
\end{align}

Если рассмотреть сумму произведений двух соседних матриц от j-1, j и j+1 внутри цепи при всевозможных значениях моментов, мы получим:

\[\sum_{\sigma = \pm 1} T(\sigma_{j-1}, \sigma_{j})T(\sigma_{j}, \sigma_{j+1}) = T^{2}(\sigma_{j-1}, \sigma_{j+1})\]

\subsection{Диагонализация Трансфер-матрицы}

Попробуем диагонализировать Трансфер-матрицу ($ T = R T^{D} R^{-1} $), тогда полное произведение матриц будет:

\[ \sum_{\sigma}\prod_{j}^{N} T(\sigma_{j}, \sigma_{j+1}) = R (T^{D})^{N} R^{-1} (\sigma_{1},\sigma_{N+1} = \sigma_{1})\]

Диагонализированная матрица будет выглядеть как:

\begin{align}
T^{D} = &
\begin{pmatrix}\label{Td}
  \lp & 0 \\
  0 & \lm
\end{pmatrix} \\ 
(T^{D})^{N} = &
\begin{pmatrix}
  \lp^{N} & 0 \\
  0 & \lm^{N}
\end{pmatrix}
\end{align}

Найдём собственные значения $\lpm$ и их собственные вектора:

\[ \lpm = e^{\bj}  \cosh{(\bh)} \pm Q\]
\[ Q = \sqrt{e^{2\bj} \cosh{(\bh)}^{2} - 2 \sinh{(2\bj)}} \]

\begin{align}\label{R}
R = &
\begin{pmatrix*}
  e^{\bj}\lpd & e^{\bj}\lmd \\
  1 & 1
\end{pmatrix*}\\
\label{RInv}
R^{-1} = &
\begin{pmatrix*}
  \frac{1}{2e^{\bj}Q} & 1-\frac{\lpd}{2Q} \\
  -\frac{1}{2e^{\bj}Q} & \frac{\lpd}{2Q}
\end{pmatrix*}
\end{align}

\[ \lpmd = e^{\bj}  \sinh{(\bh)} \pm Q\]

Эти формулы понадобятся нам позднее.

Поскольку нам нужен инвариантный след данной матрицы, т.к. матрица зависит от одного элемента, то достаточно $ Z = Tr(T^{D})^{N} $

Таким образом:

\begin{equation}\label{Zpbc}
Z\pbc = \lambda_{+}^{N} + \lambda _{-} ^{N}     
\end{equation}

\subsection{Cтатсумма цепи Изинга общего случая ($h, J \neq 0 $) : открытые гран. условия}

Расчёт статсуммы в данном случае сложнее, т.к. система не замкнута, и крайние значения не имеют внутреннего взаимодействия между с собой. Попробуем воспользоваться формулой \eqref{eq:HamTrM} с корректировкой под открытые условия:

\begin{equation}\label{eq:HamTrM2}
    H = - \frac{h}{2}\sum_{j = 1}^{N-1}  (\sigma_{j} + \sigma_{j+1}) - J \sum_{j = 1}^{N-1} \sigma_{j}  \sigma_{j+1} - \frac{h}{2}(\sigma_{1} + \sigma_{N})
\end{equation}

\begin{align*}\label{ZTrM}
    Z = \sum_{\sigma} e^{-\beta H} = \sum_{\sigma} \prod_{j=1}^{N-1} \exp{(\bj\sigma_{j}\sigma_{j+1} + \frac{1}{2}\bh(\sigma_{j} + \sigma_{j+1}))} exp(\frac{1}{2}\bh(\sigma_{1}+\sigma_{N})) = \\
    = \sum_{\sigma} \prod_{j=1}^{N-1} T(\sigma_{j}, \sigma_{j+1}) T^{'}(\sigma_{1}, \sigma_{N})
\end{align*}

Где $T'(\sigma_{1}, \sigma_{N})$ - трансфер-матрица для крайних моментов. От ранее расмотренных матриц она отличается отсутствием внутреннего взаимодействия, поэтому она представима в виде:
\begin{align*}
  T(\sigma_{1}, \sigma_{N}) = &
  \begin{pmatrix}
    \exp{(\bh)} & 1 \\
    1 & \exp{(-\bh)}
  \end{pmatrix}  
\end{align*}

К полному произведению применимы те же рассуждения, что и в периодическом случае: воспользовшись диагонализацией трансфер-матрицы T, мы получим:

\[ Z = \sum_{\sigma}\prod_{j}^{N-1} T(\sigma_{j}, \sigma_{j+1})T^{'}(\sigma_{1}, \sigma_{N}) = R (T^{D})^{N-1} R^{-1} T^{'}(\sigma_{1}, \sigma_{N})\]

Просуммировав элементы матрицы, полученной из данного произведения, мы получим:

\begin{equation}\label{Zobc}
    Z\obc = \lp^{N-1}(\frac{e^{\bj} \sinh{\bh}^2}{Q} + \frac{1}{e^{\bj}Q} +  \cosh{\bh}) - \lm^{N-1}(\frac{e^{\bj} \sinh{\bh}^2}{Q} + \frac{1}{e^{\bj}Q} -  \cosh{\bh}) 
\end{equation}

\subsection{Итоги}
Нам известна статсумма модели для общего случая: 
\[ Z\pbc = \lambda_{+}^{N} + \lambda _{-} ^{N} \]
 - для периодичного граничного условия
 
\[ Z\obc = \lp^{N-1}(\frac{e^{\bj} \sinh{\bh}^2}{Q} + \frac{1}{e^{\bj}Q} +  \cosh{\bh}) - \lm^{N-1}(\frac{e^{\bj} \sinh{\bh}^2}{Q} + \frac{1}{e^{\bj}Q} -  \cosh{\bh}) \]
- для открытого погран. случая, где 

\[ \lambda_{\pm} = e^{\bj}  \cosh{(\bh)} \pm Q\]
\[ Q = \sqrt{e^{2\bj} \cosh{(\bh)}^{2} - 2 \sinh{(2\bj)}} \]

\section{Средняя намагнинченность случая h = 0}

По предыдущим расчетам мы знаем формулу ср. намагниченности, с самого начала считая J = 0. С одной стороны, по определению среднего:

\begin{equation}\label{MeanMagnJ01}
    \la\sigma\ra = \frac{1}{ZN} \int S\ e^{-\beta H},\  H = -h\sum_{j=1}^{N}\sigma_{j} = -h S
\end{equation}

где S - сумма всех моментов в цепи. С другой стороны: 

\begin{equation}\label{MeanMagnJ02}
    \frac{1}{ZN} \int S\ e^{-\beta H} = \frac{\partial Log[Z_{J = 0}]}{\partial h} \frac{1}{\beta N} = \frac{1}{Z \beta N}  \frac{\partial Z}{\partial h} = \tanh{(\bh)}
\end{equation}

Применим эту операцию для статсуммы общего случая.
Для упрощения задачи будем рассматривать случай h = 0.

\subsection{Периодичные гран. условия}

Перед этим для простоты найдём производные всех составляющих статсумм:

\begin{equation}\label{Q'h}
    (Q)^{'}_{h} = \frac{1}{2\sqrt{e^{2\bj} \cosh{(\bh)}^{2} - 2 \sinh{(2\bj)}}}  \left( e^{2\bj}2\cosh{(\bh)} \sinh{(\bh)}  \beta\right)
\end{equation}

и при (h = 0) = 0

Тогда:

\begin{equation}\label{lpm'h}
    (\lpm)^{'}_{h} = e^{\bj}  \sinh{(\bh)} \beta \pm (Q)^{'}_{h}
\end{equation}

и при (h = 0) так же = 0

Таким образом: 

\begin{equation}\label{MeanMagnH0PBC}
    \la\sigma_{PBC}\ra = \frac{1}{Z  \beta  N}  \frac{\partial Z}{\partial h} = \frac{1}{Z  \beta  N}  \left( N  \lambda^{N-1}_{+}(\lambda_{+})^{'}_{h} +   \lambda^{N-1}_{-}(\lambda_{-})^{'}_{h}\right) = 0
\end{equation}

\subsection{Открытые гран. условия}

Найдём дополнительные значения составляющих $Z\obc$

$Q_{h = 0} = e^{-\bj}$

Также найдём значения $\lambda_{\pm}$ при h = 0
\begin{equation}\label{lpmH0}
    \lambda_{\pm (h=0)} = e^{\bj} \pm \sqrt{e^{2\bj} - \left( e^{2\bj} - e^{-2\bj} \right)} = e^{\bj} \pm e^{-\bj}
\end{equation}

Тогда $ \lambda_{+ (h = 0)} = 2\cosh{\bj} $ и $ \lambda_{- (h = 0)} = 2\sinh{\bj} $

Рассмотрим производную $Z_{OBC}$ по h, учитывая дифференцирование произведения и все полученные ранее результаты (\eqref{Q'h}, \eqref{lpm'h}, \eqref{lpmH0})

\begin{multline}\label{Zobc'h}
    \frac{\partial Z}{\partial h} = \lp^{N-1}(\frac{e^{\bj}2\sinh{\bh}\ \cosh{\bh}\ \beta Q - (Q)^{'}_{h}e^{\bj}\sinh{\bh}^{2}}{Q^{2}} - \frac{(Q)^{'}_{h}}{e^{\bj}Q^{2}} + \beta\sinh{\bh}) - 
    \\
    - \lm^{N-1}(\frac{e^{\bj}2\sinh{\bh}\ \cosh{\bh}\ \beta Q - (Q)^{'}_{h}e^{\bj}\sinh{\bh}^{2}}{Q^{2}} - \frac{(Q)^{'}_{h}}{e^{\bj}Q^{2}} - \beta\sinh{\bh}) =_{h=0} 0
\end{multline} 


\subsection{Магнитная воприимчивость}

Мы выяснили, что средняя намагниченность одномерной цепи при любом гран. условии равна нулю. Рассмотрим в таком случае магнитную восприимчивость $ X = \frac{\partial \langle m \rangle}{\partial h}$

Учитывая формулу намагниченности \eqref{MeanMagnJ01} и то, что первая производная статсуммы \eqref{Zobc'h} равна нулю:

\[ X = (\frac{1}{Z \beta} \dzdb)^{'}_{h} = \frac{1}{\beta} (\dzdb (- \frac{1}{Z^{2}} \dzdb) + \frac{1}{Z} \frac{\partial^{2} Z}{\partial h^{2}}) = \frac{1}{Z \beta} \frac{\partial^{2} Z}{\partial h^{2}}\]

После расчётов, представленных в .nb файле (раздел 21.10.2020 (поиск X)) получим:

\[ X = \frac{\beta}{2} (2Ne^{2\bj} - e^{4\bj} + 1) + \frac{\beta}{2} \tanh^{N-1}\bj (e^{4\bj} - 2 e^{2\bj} + 1)\]

Подстановка $ T = 0, \infty$ приведёт к одинаковому результату и обратной зависимости от T, что говорит о парамагнетических свойствах одномерной модели Изинга.

\section{Средняя энергия}

Чтобы удостовериться в правильности полученной формулы для статсуммы общего случая открытого гран. условия \eqref{Zobc}, проверим её на предельных условиях (h = 0, J = 0), поскольку они были рассмотрены в учебнике \cite{Swen}.

Нам известна формула средней энергии:
\begin{equation}\label{MeanE}
    \la U \ra = -\frac{\partial Log[Z]}{\partial \beta} = \frac{1}{Z} \dzdb
\end{equation}

Предварительно будет нелишним найти значения составных частей формулы и их производных по $\beta$

\[ \lambda_{\pm}^{'} = e^{\bj}J\cosh{\bh} + e^{\bj}\sinh{\bh}\ h \pm \]
\[\pm \frac{1}{Q}  (e^{\dbj}J\cosh{\bh} + e^{\dbj}\cosh{\bh}\ \sinh{\bh}\ h - \cosh{\dbj}\ 2J) \] 

В виду большого числа различных значений, составим таблицу всех составных значений в формуле.
\begin{table}[h!]\label{derLTab}
    \centering
    \begin{tabular}{c c c c c c c}
         & $\lp$ & $(\lp)^{'}_{\beta}$ & $\lm$ & $(\lm)^{'}_{\beta}$ & $Q$ & $(Q)^{'}_{\beta}$  \\ \\
        h = 0 & $2\cosh{\bj}$ & $2J\sinh{\bj}$ & $2\sinh{\bj}$ & $2J\cosh{\bj}$ & $e^{-\bj}$ & $-J e^{-\bj}$ \\
        J = 0 & $2\cosh{\bh}$ & $2h\sinh{\bh}$ & $0$ & $0$ & $\cosh{\bh}$ & $h\sinh{\bh}$\\
    \end{tabular}
    \caption{Производные составных значений статсумм}
\end{table}



\subsection{Проверка случая J = 0}

Теперь можно перейти к проверке по предельным случаям.
\[ Z\obc(h = 0) = 2^{N-1}\cosh{\bj}^{N-1} (0 + 1 + 1) - 2^{N-1}\sinh{\bj}^{N-1} (0 + 1 - 1) = 2^{N}\cosh{\bj}^{N-1}\]

\[ Z\obc(J = 0) = 2^{N-1}\cosh{\bh}^{N-1} (\frac{(\sinh{\bh})^{2} + 1}{\cosh{\bh}} +\cosh{\bh}) = \]
\[=2^{N-1}\cosh{\bh}^{N-1} (\frac{(\cosh{\bh})^{2}}{\cosh{\bh}} +\cosh{\bh}) = \]

\[= 2^{N}\cosh{\bh}^{N}\]

Как и ожидалось, статсуммы совпали с расчетами учебника \cite{Swen}, что говорит о правильности формулы. Чтобы сильнее убедиться в этом, найдём формулы средней энергии. 

Для J = 0 заранее учтём, что правое слагаемое формулы статсуммы и её производной обнулится:
\[\dzdb = \lp^{N-1}(\frac{e^{\bj}\sinh{\bh}((J\sinh{\bh} + 2h\cosh{\bh})Q - (Q)^{'}_{\beta}\sinh{\bh})}{Q^{2}} - \]
\[ - \frac{J Q + (Q)^{'}_{\beta}}{e^{\bj} Q^{2}} + h\sinh{\bh}) + \lp^{N-2} (\lp)^{'}_{\beta}(N-1)(\frac{e^{\bj} \sinh{\bh}^2}{Q} + \frac{1}{e^{\bj}Q} + \cosh{\bh})\]
При подстановке J=0 мы получим $2^{N}(\cosh{\bh})^{N-1}N\sinh{\bh}$

И в конечном счёте формула средней энергии системы при J=0: 

\[\langle U_{J = 0} \rangle = - \frac{1}{Z} \dzdb = - N h\tanh{\bh}\]

Даннная формула полностью совпадает с расчётами в учебнике \cite{Swen}, что говорит о правильности формулы для статсуммы обобщенного случая.

\subsection{Случай h = 0}

Из проделанных ранее расчётов для средней энергии системы при случае h = 0, используя соответствующую статсумму, мы получили следующую формулу:
\[ \langle U_{h = 0} \rangle = - J(N - 1)\tanh{\bj} \]
Попробуем вывести ту же формулу через статсумму общего случая.

Начнём со статсуммы:
\[ Z\obc(h=0) = \lp^{N-1}(0 + 1 + 1) - \lm^{N-1}(0 + 1 - 1) = 2^{N}(\cosh{\bj})^{N-1}\]
Поскольку формула производной статсуммы увеличится в два раза из-за ненулевых $\lm$ и $(\lm)^{'}_{\beta}$ рассмотрим их сомножители, заранее учитывая их отличие лишь в знаке правого $\cosh{\bh}$. Назовём их $A_{+}$ и $A_{-}$

Так, при подстановке в производную как $А_{+}$, так и $A_{-}$  h=0 получим ноль. А при подстановке h=0 в сами сомножители, получим:
\[A_{+(h=0)} = 2\]
\[A_{-(h=0)} = 0\]
Таким образом, наша формула $\dzdb_{h=0}$ сократилась в 4 раза и равна:
\[ \dzdb_{h=0} = (N-1)\lp^{N-2}(\lp)^{'}_{\beta}2 = J(N-1)2^{N}(\cosh{\bj})^{N-1}\sinh{\bj}\]

Итоговая формула средней энергии будет:
\[ \langle U_{h=0} \rangle = - \frac{1}{Z} \dzdb = - J (N - 1) \tanh{\bj}\]

\subsection{Сравнение средней энергии моделей с периодичным и открытым гран. условиями}

Найдём формулу средней энергии для случая с периодичным гран. условием для h = 0. Воспользовавшись формулой \eqref{Zpbc} для нахождения средней энергии через \eqref{MeanE} и таблицей производных, получим:

\begin{align*}
    \la E_{PBC (h = 0)} \ra = \frac{1}{Z\pbc(h = 0)}\left(N\lp^{N-1}(\lp)\prpb + N\lm^{N-1}(\lm)\prpb \right) = \\
    = JN2^{N}\sinh{\bj}\ \cosh{\bj} \frac{(\cosh{\bj})^{N-2} + (\sinh{\bj})^{N-2}}{(\cosh{\bj})^{N} + (\sinh{\bj})^{N}} = \\
    = JN2^{N} \tanh{\bj} \frac{1 + (\tanh{\bj})^{N-2}}{1 + (\tanh{\bj})^{N}} \approx \footnotemark JN2^{N} \tanh{\bj}
\end{align*}

\footnotetext{\begin{align*}
        \frac{1+(\tanh{x})^{N-2}}{1+(\tanh{x})^{N}} = \frac{1+x^{N}(\dfrac{1}{x^{2}} + (\dfrac{2}{3} - \dfrac{n}{3}) + O(x))}{1+x^{N}(1 - \dfrac{nx^{2}}{3} + O(x^{3}))} \approx 1 & ,\ x \to 0,\ \ \ \  1, x \to \infty
    \end{align*}}

\section{Теплоёмкость на спин при h = 0}
Теперь, поскольку наша формула статсуммы $Z_{OBC}$ (для крайних случаев) и её производная (для средних наблюдаемых) полностью верна, проверим правильность статсуммы до второй производной по $\beta$ для нахождения темплоёмкости на спин С в случае нулевого поля.

\subsection{Открытое гран. условие}

Из учебника данная формула выглядит следующим образом:
\[ c = \frac{1}{N} \frac{\partial U}{\partial T} = - \frac{1}{N k_{B} T^{2}} \frac{\partial U}{\partial \beta} \approx k_{B} \beta^{2} J^{2} (sech \bj)^{2} \]

Предыдущие вычисления уже показали правильность формулы средней энергии, однако для более полной проверки выразим U через статсумму, и следовательно:

\[ -\frac{1}{N k_{B} T^{2}} \frac{\partial U}{\partial \beta} = - k_{B} \beta^{2} \frac{1}{N} \frac{\partial}{\partial \beta} (- \frac{1}{Z} \dzdb) = k_{B} \beta^{2} \frac{1}{N} (- \frac{1}{Z^{2}} (\dzdb)^{2} + \frac{1}{Z} \frac{\partial^{2} Z}{\partial \beta^{2}}) \]

Теперь для определения второйй производной статсуммы перейдём к той же замене, как в конце предыдущего раздела:

\[ Z\obc = \lp^{N-1} A_{+} - \lm^{N-1} A_{-} \]
\[ (Z\obc)\prpb = (N-1) \lp^{N-2} (\lp)\prpb A_{+} + \lp^{N-1} (A_{+})\prpb - (N-1) \lm^{N-2} (\lm)\prpb A_{-} - \lm^{N-1} (A_{-})\prpb \]

Т.к. мы знаем, что первые производные $(A_{\pm})\prpb = 0$ и $A_{-} = 0, A_{+} = 2$, то половина второй производной (вследствие производной произведения) обнулится. Будет лучше заранее найти значения вторых производных А и $\lpm$ при h = 0.

\[ (A_{\pm})\vprpb =_{h=0} 0 \]
\[ (\lpm)\vprpb =_{h=0} J^{2}(e^{\bj} \pm e^{-\bj}) \]

Таким образом, единственным необнулённым слагаемым второй производной будет первое и:
\[ Z\obc = 2\lp^{N-1} \]
\[ (Z\obc)\prpb = 2(N-1) \lp^{N-2} (\lp)\prpb\]
\[ (Z\obc)\vprpb = 2(N-1)((N-2)\lp^{N-3}(\lp)\prpb ^{2} + \lp^{N-2}(\lp)\vprpb) \]

Раскрыв все $\lp$ и подставив в формулу теплоёмкости на спин, получим:
\[ c = k_{B} \beta^{2} (1 - \frac{1}{N}) (- (N-1) (\frac{(\lp)\prpb}{\lp})^{2} + (N-2) (\frac{(\lp)\prpb}{\lp})^{2} + \frac{(\lp)\vprpb}{\lp}) =\]
\[ = k_{B} \beta^{2} J^{2} (1 - \frac{1}{N}) (1 - (\frac{\sinh{\bj}}{\cosh{\bj}})^{2}) \approx k_{B} \beta^{2} J^{2} (sech \bj)^{2} \]

Формулы полностью совпали.

\subsection{Периодическое гран. условие}

Формула теплоёмкости на спин для данного условия отсутствует в учебнике, поэтому сравнить полученный результат с первоисточником не получится и к вычислениям данной формулы требуется особое внимание.

Начнём с формулы статсуммы:
\[ Z\pbc = \lambda_{+}
 ^{N} + \lambda _{-} ^{N} = \lp^{N}(1 + (\frac{\lp}{\lm})^{N}) =_{h=0} 2^{N} (\cosh{\bj})^{N} (1 + (\tanh{\bj})^{N}) \]
 
 \[ (Z\pbc)\prpb = N(\lp^{N-1} (\lp)\prpb + \lm^{N-1} (\lm)\prpb) = J N 2^{N} (\cosh{\bj})^{N-1} \sinh{\bj} (1 + (\tanh{\bj})^{N-2}) \]
 
 \[ (Z\pbc)\vprpb = N (\lp^{N-1} (\lp)\vprpb) + (N-1)\lp^{N-2} (\lp)\prpb^{2} + \lm^{N-1} (\lm)\vprpb) + (N-1)\lm^{N-2} (\lm)\prpb^{2}) = \]
 
 \[ = 2^{N} N J^{2} (\cosh{\bj})^{N} (1 + (N-1)(\tanh{\bj})^{2} + (N-1)(\tanh{\bj})^{N-2} + \tanh{\bj})\]
 
Прошлые расчёты показали, что формула теплоёмкости на спин выражается через статсумму как:

\[ c = \frac{k_{B} \beta^{2}}{N} (- \frac{1}{Z^{2}} (\dzdb)^{2} + \frac{1}{Z} \frac{\partial^{2} Z}{\partial \beta^{2}}) \]

В таком случае, при подстановке статсуммы и производных, мы получим:

\[ c = k_{B} \beta^{2} J^{2} \left(1 + (N-1) \tanh^{2}\bj (\frac{1 + \tanh^{N-4}\bj}{1 + \tanh^{N}\bj}) - N \tanh^{2}\bj (\frac{1 + \tanh^{N-2}\bj}{1 + \tanh^{N}\bj})^{2}\right) \]

В случае термодинамического предела, все дроби вида $ \frac{1 + \tanh}{1 + \tanh}$ стремятся к единице (есть небольшое отклонение, которое при увеличении N смещается вправо и одновременно уменьшается. Тогда в итоге:

\[ c = k_{B} \beta^{2} J^{2} sech^{2} \bj \]

\section{Свободная энергия}
Учитывая предыдущие вычисления, будет удобно проверить формулу другой величины - свободной энергии дл 
я случая h = 0. Для этого слегка преобразуем нашу статсумму:
\[ Z\obc = \lp^{N-1}A_{+}(1 - (\frac{\lm}{\lp})^{N-1}(\frac{A_{-}}{A_{+}}) \]

где \[ A_{+} = \frac{e^{\bj} \sinh{\bh}^2}{Q} + \frac{1}{e^{\bj}Q} +  \cosh{\bh}\]
\[ A_{-} = \frac{e^{\bj} \sinh{\bh}^2}{Q} + \frac{1}{e^{\bj}Q} -  \cosh{\bh}\]

Тогда свободная энергия для случая h=0 будет равна:
\[ F_{h=0} = k_{B}T\ln{Z} =  k_{B}(N-1)\ln{\lp} + k_{B}T\ln{A+} + k_{B}T\ln{(1 - (\frac{\lm}{\lp})^{N-1}(\frac{A_{-}}{A_{+}}))}\]

Ранее мы узнали все преобразования при h = 0: $A_{+} = 2, A_{-} = 0$, следовательно:
\[ F_{h=0} = k_{B}T(N-1)\ln{(2\cosh{\bj})} + k_{B}T\ln{2}\]

Результаты снова совпали с формулой из учебника \cite{Swen}.

Тогда руководствуясь предыдущими расчётами для случая J=0, свободная энергия для данного случая (зная, что $\lp = 2\cosh{\bh}, \lm = 0, A_{+} = 2\cosh{\bh}$) будет равна:
\[ F_{J=0} = k_{B}T N \ln{(2\cosh{\bh})}\]
\section{Разница между открытым и периодичным случаем}

Будем рассматривать разность между различными энергетическими величинами при разных случаев.

\subsection{Средняя энергия системы}


Найдём разность средней энергии открытого и периодичного случая:

\[ \la U_{OBC} \ra - \la U_{OBC} \ra = -\frac{\partial Log[Z_{OBC}]}{\partial \beta} + \frac{\partial Log[Z_{PBC}]}{\partial \beta} = -\frac{\partial Log[\frac{Z_{OBC}}{Z_{PBC}}]}{\partial \beta} \]

\[ \frac{Z_{OBC}}{Z_{PBC}} = \frac{A_{+}\left( 1 + (\frac{\lambda_{-}}{\lambda_{+}})^{N - 1}  \frac{A_{-}}{A_{+}}\right)}{\lambda{+}(1 + (\frac{\lambda_{-}}{\lambda_{+}})^{N})} \]

Учитывая что мы рассматриваем системы при $N \rightarrow \infty$,
все скобки вида $ Log[ 1 + (<1)^{N} ] \approx (<1)^{N} $ 

Тогда

\[ \la U_{OBC} \ra - \la U_{PBC} \ra =  -\frac{\partial ( Log[\frac{Z_{OBC}}{Z_{PBC}}])}{\partial \beta} = -\frac{\partial Log[\frac{A_{+}}{\lambda{+}}] + O((\frac{\lambda_{-}}{\lambda_{+}})^{N - 1})}{\partial \beta}\]

Здесь не имеет значения рассматривать правое слагаемое, т.к. при $N \to \infty$ оно останется бесконечно малой. Тогда формула примет вид 

\begin{equation}\label{diffMeanE1}
    \la U_{OBC} \ra - \la U_{PBC} \ra = \frac{(\ap)\prpb}{\ap} - \frac{(\lp)\prpb}{\lp}
\end{equation}

Рассмотрим все значения и значения производных по $\beta$ $ \lambda_{+}$ и $A_{+}$ при h = 0 и J = 0 из таблицы.

Путём подстановки в полученную формулу производной \eqref{diffMeanE1}, получим:

При $h = 0:$ = -J

При $J = 0:$ = 0 

\begin{thebibliography}{9}
\bibitem[1]{Swen} Swendsen R. An introduction to statistical mechanics and thermodynamics. – Oxford University Press, USA, 2020.
\end{thebibliography}

\end{document}
