\section{Введение}

В рамках проекта 19455 "Решёточные модели макромолекул", где одной в основных задач было исследование долей мономеров случайного блуждания без самопересечений (далее SAW) фиксированной длины на квадратной решетке, возник вопрос сравнения значений долей для конформаций в пределе бесконечной длины $n_i$ и вероятность блуждания иметь атмосферу k $p^{(k)}$ (то есть, вероятность, что конец блуждания фиксированной длины $N$ будет иметь k свободных от мономеров узлов решётки, или k возможных способов присоединить новый, $N+1$-й узел). 
Предполагалось, что доля узлов с i соседями распространяется и на концы блужданий - то есть, если блуждание имеет возможность присоединить новый узел, и распределение атмосфер имеет вид $ P = (p^{(1)},p^{(2)},p^{(3)}) / (p^{(1)}+p^{(2)}+p^{(3)})$, и распределение долей узлов с 4,3,2 соседями $n = (n_4, n_3, n_2)$, то в пределе на бесконечную длину блуждания $P = n$. 

Доли узлов с фикс. числом соедей для модели SAW подсчитывали симуляциями методами Монте-Карло, в то время как данные о распределении атмосфер были взяты из статьи Томаса Преллберга \cite{Spitser1969}.
Результатом исследования оказалось близкое совпадение $p_3$ и $n_2$, однако разница всех сопоставленных значених оказались больше погрешности, что означало, что обе величины оценивают разные аспекты поведения модели блуждания без самопересечений.

Данная работа посвящена схожему исследованию, но уже для родительской по отношешию к SAW модели - простого случайного блуждания (далее Rand-Walk). Для этой модели узлы решетки могут быть посещены неограниченное количество раз. Подсчёт долей узлов проводился среди уникальных узлов случайного блуждания, дабы предупредить пересчёт. Полный алгоритм расписан в отчёте проекта 19455.

Основной же целью практики будет исследование величины, предположительно являющейся аналогом атмосфер Преллберга, представленная в качестве ответа задачи III.9 в книге "Теория случайных блужданий" Франка Спицера. Задача сформулирована следующим образом: 

\begin{itemize}
    \item Случайное блуждание на квадратной решётке начинается начинается из некоторой точки $x_0 = \chi$, не лежащей в начале координат.
    \item Процесс случайного блуждания длится не фиксированное количество шагов, а до фиксированной \textit{точки остановки} - до достижения блужданием начала коордиинат $x_{end} = 0$
    \item До достижения точки остановки блуждание может посетить одну или несколько соседних с началом координат точек - (0,1), (1,0), (0,-1), (-1,0). Пусть число посещенных блужданием соседних точек  $N \in \{1, 2, 3, 4\}$
    \item Задачей является вычислить вероятности блуждания посетить каждое возможное количество соседних точек для бесконечно удаленной от начала координат начальной точки блуждания $\chi$:
    
    \[ p_{n} = \lim_{|\chi|\to \infty} P_{\chi}[N = n] = ?,\ \ \ n = 1, 2, 3, 4\]
\end{itemize}

Так же в качестве подсказки было указано, что отношение $p_1:p_2:p_3:p_4$ почти равно $4:3:2:1$.

Задачами научно-исследовательской практики являются проработка материала книги, теория которой позволит решить задачу III.9 и получить её теоретическое решение, поиск экспериментального ответа на задачу путём симуляцией модели, представленной в задаче методами Монте-Карло, а так же сравнение результатов с долями узлов модели, исследованной ранее в рамках проекта.